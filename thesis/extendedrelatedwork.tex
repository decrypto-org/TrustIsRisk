\documentclass[11pt]{llncs}
\usepackage{preamble}

\begin{document}
  Trust is a wide topic that exhibits very interesting properties and can be defined in several, often competing manners.
  Here we will present briefly several alternative approaches that have been followed in pursuit of a satisfactory model of
  trust and another tightly related and equally elusive concept, reputation.

  Mui and Halberstadt \cite{mui} have proposed an elaborate model based on the triptych "trust, reciprocity, reputation",
  where reciprocal actions of an agent $A$ generate a corresponding reputation, which in turn influences other agents' trust
  to $A$. Trusting $A$ inspires other agents to reciprocate, thus completing the cycle. In this model, actions are limited to
  $cooperate$ and $defect$, reciprocity and reputation between two agents are real numbers in $\left[0, 1\right]$, the latter
  also depending on the context of interest. Lastly trust is derived as a mean value based on the agent's reputation and the
  known history. The variables are connected using the Beta distribution from statistics.

  This model has little resemblance with Trust Is Risk not only in the formalities, but mainly in the approach taken. Trust
  Is Risk proposes a new financial game, whereas \cite{mui} attempts to model and predict all kinds of conceivable trust.
  Trust Is Risk does not use statistics nor scales trust to $\left[0, 1\right]$ and thus can provide strong results, such as
  the Risk Invariance theorem.

  FIRE \cite{fire} constitutes another attempt to tackle trust, this time in a practical setting. FIRE aims to create a
  decentralized rating system for services provided. It essentially calculates trust as "the sum of all the available ratings
  weighted by the rating relevance and normalized to the range of $\left[-1, 1\right]$." This setup needs two very disputable
  assumptions: Firstly that "[a]gents are willing to share their experiences with others" and secondly that "[a]gents are
  honest in exchanging information with one another." Trust Is Risk does not make these assumptions, but can function even
  when each player follows any strategy she desires.

  
 % \import{thesis/}{riskinvalgsintro.tex}
\end{document}
