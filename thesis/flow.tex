  \section{Trust Flow}
    We can now define the indirect trust, or simply trust, from $A$ to $B$.
    \begin{definition}[Indirect Trust]
       The indirect trust from $A$ to $B$ after turn $j$ is defined as the maximum possible value that can be stolen from
       $A$ after turn $j$ in the setting of \texttt{TransitiveGame(}$\mathcal{G}_j$\texttt{,}$A$\texttt{,}$B$\texttt{)}.
    \end{definition}
    \noindent It is $Tr_{A \rightarrow B} \geq DTr_{A \rightarrow B}$. The next theorem shows that
    $Tr_{A \rightarrow B}$ is finite.
    \begin{theorem}[Trust Convergence Theorem] \ \\
       \label{convergence}
       Consider a Transitive Game. There exists a turn such that all subsequent turns are empty.
%       \begin{equation*}
%          \forall j \geq j', Turn_j = \emptyset \enspace.
%       \end{equation*}
    \end{theorem}
    \begin{proofsketch}
       If the game didn't converge, the $Steal\left(\right)$ actions would continue forever without reduction of the amount
       stolen over time, thus they would reach infinity. However this is impossible, since there exists only finite total
       trust.
    \end{proofsketch}
    Full proofs of all theorems and lemmas can be found in the Appendix.

    In the setting of \texttt{TransitiveGame(}$\mathcal{G}$\texttt{,}$A$\texttt{,}$E$\texttt{)}, we make use of the notation
    $Loss_A = Loss_{A, j}$, where $j$ is a turn that the game has converged. It is important to note that $Loss_A$ is
    not the same for repeated executions of this kind of game, since the order in which players are chosen may differ between
    executions and the conservative players are free to choose which incoming trusts they will steal and how much from each.

    Let $G$ be a weighted directed graph. We will investigate the maximum flow on this graph. For an introduction to the
    maximum flow problem see \cite{clrs} p. 708. Considering each edge's capacity as its weight, a flow assignment
    $X = [x_{vw}]_{V \times V}$ with a source $A$ and a sink $B$ is valid when:
    \begin{equation}
    \label{flow1}
       \forall (v, w) \in E, x_{vw} \leq c_{vw} \mbox{ and}
    \end{equation}
    \begin{equation}
    \label{flow2}
       \forall v \in V \setminus \{A,B\}, \sum\limits_{w \in N^{+}(v)}x_{wv} = \sum\limits_{w \in N^{-}(v)}x_{vw}
       \enspace.
    \end{equation}
    We do not suppose any skew symmetry in $X$. The flow value is $\sum\limits_{v \in N^{+}\left(A\right)}x_{Av}$, which is
    proven to be equal to $\sum\limits_{v \in N^{-}\left(B\right)}x_{vB}$. There exists an algorithm that returns the maximum
    possible flow from $A$ to $B$, namely $MaxFlow\left(A, B\right)$. This algorithm evidently needs full knowledge of the
    graph. The fastest version of this algorithm runs in $O\left(|V||E|\right)$ time \cite{maxflownm}. We refer to the flow
    value of $MaxFlow\left(A, B\right)$ as $maxFlow\left(A, B\right)$.

    We will now introduce two lemmas that will be used to prove the one of the central results of this work, the Trust Flow
    theorem.
    \begin{lemma}[MaxFlows Are Transitive Games] \ \\
       \label{maxflowgame}
       Let $\mathcal{G}$ be a game graph, let $A, E \in \mathcal{V}$ and $MaxFlow\left(A, E\right)$ the maximum flow from
       $A$ to $E$ executed on $\mathcal{G}$. There exists an execution of
       \texttt{TransitiveGame(}$\mathcal{G}, A, E$\texttt{)} such that $maxFlow\left(A, E\right) \leq Loss_A$.
%       \begin{equation*}
%          maxFlow\left(A, E\right) \leq Loss_A \enspace.
%       \end{equation*}
    \end{lemma}
    \begin{proofsketch}
       The desired execution of \texttt{TransitiveGame()} will contain all flows from the $MaxFlow\left(A, E\right)$ as
       equivalent $Steal\left(\right)$ actions. The players will play in turns, moving from $E$ back to $A$. Each player will
       steal from his predecessors as much as was stolen from her. The flows and the conservative strategy share the property
       that the total input is equal to the total output.
    \end{proofsketch}
    \begin{lemma}[Transitive Games Are Flows] \ \\
       \label{gameflow}
       Let $\mathcal{H} = $\texttt{TransitiveGame(}$\mathcal{G}, A, E$\texttt{)} for some game graph $\mathcal{G}$ and $A,
       E \in \mathcal{V}$. There exists a valid flow
       $X = \{x_{wv}\}_{\mathcal{V} \times \mathcal{V}}$ on $\mathcal{G}_0$ such that
       $\sum\limits_{v \in \mathcal{V}}x_{Av} = Loss_A$.
%       \begin{equation*}
%          \sum\limits_{v \in \mathcal{V}}x_{Av} = Loss_A \enspace.
%       \end{equation*}
    \end{lemma}
    \begin{proofsketch}
       If we exclude the sad players from the game, the $Steal\left(\right)$ actions that remain constitute a valid flow from
       $A$ to $E$.
    \end{proofsketch}
    \begin{theorem}[Trust Flow Theorem] \ \\
       \label{trustflow}
       Let $\mathcal{G}$ be a game graph and $A, E \in \mathcal{V}$. It holds that
       \begin{equation*}
          Tr_{A \rightarrow E} = maxFlow\left(A, E\right) \enspace.
       \end{equation*}
    \end{theorem}
    \begin{proof}%[Trust Flow Theorem (\ref{trustflow})] \ \\
       From lemma (\ref{maxflowgame}) there exists an execution of the Transitive Game such that
       $Loss_A \geq maxFlow\left(A, E\right)$.
       Since $Tr_{A \rightarrow E}$ is the maximum loss that $A$ can suffer after the convergence of the Transitive Game, we
       see that
       \begin{equation}
       \label{trgeqmaxflow}
          Tr_{A \rightarrow E} \geq maxFlow\left(A, E\right) \enspace.
       \end{equation}
       But some execution of the Transitive Game gives $Tr_{A \rightarrow E} = Loss_A$.
       From lemma (\ref{gameflow}), this execution corresponds to a flow. Thus
       \begin{equation}
       \label{trleqmaxflow}
          Tr_{A \rightarrow E} \leq maxFlow\left(A, E\right) \enspace.
       \end{equation}
       The theorem follows from (\ref{trgeqmaxflow}) and (\ref{trleqmaxflow}).
    \end{proof}

%    \begin{proofsketch}
%       The theorem follows directly from lemma \ref{maxflowgame} and \ref{gameflow}.
%    \end{proofsketch}
     Note that the maxFlow is the same in the following two cases: If a player chooses the evil strategy and if that player
     chooses a variation of the evil strategy where she does not nullify her outgoing direct trust.

     Here we see another important theorem that gives the basis for risk-invariant transactions between different, possibly
     unknown, parties.
     \begin{theorem}[Risk Invariance Theorem]
     \label{riskinv}
        Let $\mathcal{G}$ game graph, $A, B \in \mathcal{V}$ and $l$ the desired value to be transferred from $A$ to $B$,
        with $l \leq Tr_{A \rightarrow B}$. Let also $\mathcal{G}'$ with the same nodes as $\mathcal{G}$ such that
        \begin{equation*}
           \forall v \in \mathcal{V}' \setminus \{A\}, \forall w \in \mathcal{V}', DTr'_{v \rightarrow w} =
           DTr_{v \rightarrow w} \enspace.
        \end{equation*}
        Furthermore, suppose that there exists an assignment for the outgoing trust of $A, DTr'_{A \rightarrow v}$, such that
        \begin{equation}
        \label{primetrust}
           Tr'_{A \rightarrow B} = Tr_{A \rightarrow B} - l \enspace.
        \end{equation}
        Let another game graph, $\mathcal{G}''$, be identical to $\mathcal{G}'$ except for the following change:
        \begin{equation*}
           DTr''_{A \rightarrow B} = DTr'_{A \rightarrow B} + l \enspace.
        \end{equation*}
        It then holds that
        \begin{equation*}
           Tr''_{A \rightarrow B} = Tr_{A \rightarrow B} \enspace.
        \end{equation*}
     \end{theorem}
     \begin{proof}
        The two graphs $\mathcal{G}'$ and $\mathcal{G}''$ differ only on the weight of the edge $\left(A, B\right)$, which is
        larger by $l$ in $\mathcal{G}''$. Thus the two $MaxFlow$s will choose the same flow, except for $\left(A, B\right)$,
        where it will be $x''_{AB} = x'_{AB} + l$.
     \end{proof}
     It is intuitively obvious that it is possible for $A$ to reduce her outgoing direct trust in a manner that achieves
     (\ref{primetrust}), since $maxFlow\left(A, B\right)$ is continuous with respect to $A$'s outgoing direct trusts. We
     leave this calculation as part of further research.
