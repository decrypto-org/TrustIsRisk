    \subsection{Proofs, Lemmas and Theorems}
    \begin{lemma}[$Loss$ Equivalent to $Damage$] \ \\
       Consider a Transitive Game. Let $j \in \mathbb{N}$ and $v = Player\left(j\right)$ such that $v$ is following the
       conservative strategy. It holds that
       \begin{equation*}
          \min\left(in_{v, j}, Loss_{v, j}\right) = \min\left(in_{v, j}, Damage_{v, j}\right) \enspace.
       \end{equation*}
    \end{lemma}
    \begin{proof} \ \\
          \textbf{Case 1:} Let $v \in Happy_{j-1}$. Then
          \begin{enumerate}
             \item $v \in Happy_j$ because $Turn_{j} = \emptyset$,
             \item $Loss_{v, j} = 0$ because otherwise $v \notin Happy_j$,
             \item $Damage_{v, j} = 0$, or else any reduction in direct trust to $v$ would increase equally
             $Loss_{v, j}$ (line~\ref{trsteallossincrease}), which cannot be decreased again but during an Angry player's turn
             (line~\ref{trsteallossdecrease}).
             \item $in_{v, j} \geq 0$
          \end{enumerate}
          Thus
          \begin{equation*}
             \min\left(in_{v, j}, Loss_{v,j}\right) = \min\left(in_{v, j}, Damage_{v,j}\right) = 0 \enspace.
          \end{equation*}
          \textbf{Case 2:} Let $v \in Sad_{j-1}$. Then
          \begin{enumerate}
             \item $v \in Sad_j$ because $Turn_{j} = \emptyset$, 
             \item $in_{v, j} = 0$ (line~\ref{trstealifentersad}),
             \item $Damage_{v, j} \geq 0 \wedge Loss_{v, j} \geq 0$.
          \end{enumerate}
          Thus
          \begin{equation*}
             \min\left(in_{v, j}, Loss_{v,j}\right) = \min\left(in_{v, j}, Damage_{v,j}\right) = 0 \enspace.
          \end{equation*}
          If $v \in Angry_{j-1}$ then the same argument as in cases 1 and 2 hold when $v \in Happy_j$ and $v \in Sad_j$
          respectively if we ignore the argument (1). Thus the theorem holds in every case.
    \end{proof}

    \begin{sepproof}{Proof of Theorem \ref{convergence}: Trust Convergence} \ \\
    \label{convergenceproof}
       First of all, after turn $j_0$ player $E$ will always pass her turn
       because she has already nullified her incoming and outgoing direct trusts in $Turn_{j_0}$, the evil strategy does not
       contain any case where direct trust is increased or where the evil player starts directly trusting another player and
       the other players do not follow a strategy in which they can choose to $Add\left(\right)$ trust to $E$. The same holds
       for player $A$ because she follows the idle strategy. As far as the rest of the players are concerned, consider the
       Transitive Game. As we can see from lines~\ref{trsteallossinit} and~\ref{trsteallossincrease}
       -~\ref{trsteallossdecrease}, it is
       \begin{equation*}
          \forall j, \sum\limits_{v \in \mathcal{V}_j}Loss_v = in_{E, j_0-1} \enspace.
       \end{equation*}
       In other words, the total loss is constant and equal to the total value stolen by $E$. Also, as we can see in
       lines~\ref{trstealsadinit} and~\ref{trstealtrueentersad}, which are the only lines where the $Sad$ set is modified,
       once a player enters the $Sad$ set, it is impossible to exit from this set. Also, we can see that players in $Sad
       \cup Happy$ always pass their turn. We will now show that eventually the $Angry$ set will be empty, or equivalently
       that eventually every player will pass their turn. Suppose that it is possible to have an infinite amount of turns
       in which players do not choose to pass. We know that the number of nodes is finite, thus this is possible only if
       \begin{equation*}
          \exists j': \forall j \geq j', |Angry_j \cup Happy_j| = c > 0 \wedge Angry_j \neq \emptyset \enspace.
       \end{equation*}
       This statement is valid because the total number of angry and happy players cannot increase because no player leaves
       the $Sad$ set and if it were to be decreased, it would eventually reach 0. Since $Angry_j \neq \emptyset$, a player
       $v$ that will not pass her turn will eventually be chosen to play. According to the Transitive Game, $v$ will either
       deplete her incoming trust and enter the $Sad$ set (line~\ref{trstealtrueentersad}), which is contradicting $|Angry_j
       \cup Happy_j| = c$, or will steal enough value to enter the $Happy$ set, that is $v$ will achieve $Loss_{v, j} = 0$.
       Suppose that she has stolen $m$ players. They, in their turn, will steal total value at least equal to the value
       stolen by $v$ (since they cannot go sad, as explained above). However, this means that, since the total value being
       stolen will never be reduced and the turns this will happen are infinite, the players must steal an infinite amount of
       value, which is impossible because the direct trusts are finite in number and in value. More precisely, let $j_1$ be
       a turn in which a conservative player is chosen and
       \begin{equation*}
          \forall j \in \mathbb{N}, DTr_j = \sum\limits_{w,w' \in \mathcal{V}}DTr_{w \rightarrow w', j} \enspace.
       \end{equation*}
       Also, without loss of generality, suppose that
       \begin{equation*}
          \forall j \geq j_1, out_{A, j} = out_{A, j_1} \enspace.
       \end{equation*}
       In $Turn_{j_1}$, $v$ steals
       \begin{equation*}
          St = \sum\limits_{i=1}^{m}y_i \enspace.
       \end{equation*}
       We will show using induction that
       \begin{equation*}
          \forall n \in \mathbb{N}, \exists j_n \in \mathbb{N} : DTr_{j_n} \leq DTr_{j_1-1} - nSt \enspace.
       \end{equation*}

       Base case: It holds that
       \begin{equation*}
          DTr_{j_1} = DTr_{j_1-1} - St \enspace.
       \end{equation*}
       Eventually there is a turn $j_2$ when every player in $N^{-}(v)_{j-1}$ will have played. Then it holds that
       \begin{equation*}
          DTr_{j_2} \leq DTr_{j_1} - St = DTr_{j_1-1} - 2St \enspace,
       \end{equation*}
       since all players in $N^{-}(v)_{j-1}$ follow the conservative strategy, except for $A$, who will not have been stolen
       anything due to the supposition.

       Induction hypothesis: Suppose that
       \begin{equation*}
          \exists k > 1 : j_k > j_{k-1} > j_1 \Rightarrow DTr_{j_k} \leq DTr_{j_{k-1}} - St \enspace.
       \end{equation*}

       Induction step: There exists a subset of the $Angry$ players, $S$, that have been stolen at least value $St$ in total
       between the turns $j_{k-1}$ and $j_k$, thus there exists a turn $j_{k+1}$ such that all players in $S$ will have
       played and thus
       \begin{equation*}
          DTr_{j_{k+1}} \leq DTr_{j_k} - St \enspace.
       \end{equation*}
       We have proven by induction that
       \begin{equation*}
          \forall n \in \mathbb{N}, \exists j_n \in \mathbb{N} : DTr_{j_n} \leq DTr_{j_1-1} - nSt \enspace.
       \end{equation*}
       However
       \begin{equation*}
          DTr_{j_1-1} \geq 0 \wedge St > 0 \enspace,
       \end{equation*}
       thus
       \begin{equation*}
          \exists n' \in \mathbb{N} : n'St > DTr_{j_1-1} \Rightarrow DTr_{j_{n'}} < 0 \enspace.
       \end{equation*}
       We have a contradiction because
       \begin{equation*}
          \forall w,w' \in \mathcal{V}, \forall j \in \mathbb{N}, DTr_{w \rightarrow w', j} \geq 0 \enspace,
       \end{equation*}
       thus eventually $Angry = \emptyset$ and everybody passes.
    \end{sepproof}

    \begin{sepproof}{Proof of Lemma \ref{maxflowgame}: MaxFlows Are Transitive Games} \ \\
    \label{maxflowgameproof}
       We suppose that the turn of $\mathcal{G}$ is 0. In other words, $\mathcal{G} = \mathcal{G}_0$. Let
       $X = \{x_{vw}\}_{\mathcal{V} \times \mathcal{V}}$ be the flows returned by $MaxFlow\left(A, E\right)$. For any graph
       $G$ there exists a $MaxFlow$ that is a DAG. We can easily prove this using the Flow Decomposition theorem
       \cite{amo}, which states that each flow can be seen as a finite set of paths from $A$ to $E$ and cycles, each
       having a certain flow. We execute $MaxFlow\left(A, E\right)$ and we apply the aforementioned theorem. The
       cycles do not influence the $maxFlow\left(A, E\right)$, thus we can remove these flows. The resulting flow is a
       $MaxFlow\left(A, E\right)$ without cycles, thus it is a DAG. Topologically sorting this DAG, we obtain a total order
       of its nodes such that $\forall$ nodes $v, w \in \mathcal{V} : v < w \Rightarrow x_{wv} = 0$ \cite{clrs}. Put
       differently, there is no flow from larger to smaller nodes. $E$ is maximum since it is the sink and thus has no
       outgoing flow to any node and $A$ is minimum since it is the source and thus has no incoming flow from any node. The
       desired execution of Transitive Game will choose players following the total order inversely, starting from player
       $E$. We observe that $\forall v \in \mathcal{V} \setminus \{A, E\}, \sum\limits_{w \in \mathcal{V}}x_{wv} =
       \sum\limits_{w \in \mathcal{V}}x_{vw} \leq maxFlow\left(A, E\right) \leq in_{E, 0}$. Player $E$ will follow a modified
       evil strategy where she steals value equal to her total incoming flow, not her total incoming trust. Let $j_2$ be the
       first turn when $A$ is chosen to play. We will show using strong induction that there exists a set of valid actions
       for each player according to their respective strategy such that at the end of each turn $j$ the corresponding player
       $v = Player\left(j\right)$ will have stolen value $x_{wv}$ from each in-neighbour $w$.

       Base case: In turn 1, $E$ steals value equal to $\sum\limits_{w \in \mathcal{V}}x_{wE}$, following the modified evil
       strategy.
       \begin{equation*}
          Turn_1 = \bigcup\limits_{v \in N^{-}\left(E\right)_0}\{Steal\left(x_{vE}, v\right)\}
       \end{equation*}

       Induction hypothesis: Let $k \in [j_2 - 2]$. We suppose that $\forall i \in [k]$, there exists a valid set of actions,
       $Turn_i$, performed by $v = Player\left(i\right)$ such that $v$ steals from each player $w$ value equal to $x_{wv}$.
       \begin{equation*}
          \forall i \in [k], Turn_i = \bigcup\limits_{w \in N^{-}\left(v\right)_{i-1}}\{Steal\left(x_{wv}, w\right)\}
       \end{equation*}

       Induction step: Let $j = k + 1, v = Player\left(j\right)$. Since all the players that are greater than $v$ in the
       total order have already played and all of them have stolen value equal to their incoming flow, we deduce that $v$ has
       been stolen value equal to $\sum\limits_{w \in N^{+}\left(v\right)_{j-1}}x_{vw}$. Since it is the first time $v$
       plays, $\forall w \in N^{-}\left(v\right)_{j-1}, DTr_{w \rightarrow v, j-1} = DTr_{w \rightarrow v, 0} \geq x_{wv}$, thus
       $v$ is able to choose the following turn:
       \begin{equation*}
          Turn_j = \bigcup\limits_{w \in N^{-}\left(v\right)_{j-1}}\{Steal\left(x_{wv}, w\right)\}
       \end{equation*}
       Moreover, this turn satisfies the conservative strategy since
       \begin{equation*}
          \sum\limits_{w \in N^{-}\left(v\right)_{j-1}}x_{wv} = \sum\limits_{w \in N^{+}\left(v\right)_{j-1}}x_{vw} \enspace.
       \end{equation*}
       Thus $Turn_j$ is a valid turn for the conservative player $v$.

       We have proven that in the end of turn $j_2 - 1$, player $E$ and all the conservative players will have stolen value
       exactly equal to their total incoming flow, thus $A$ will have been stolen value equal to her outgoing flow, which is
       $maxFlow(A, E)$. Since there remains no Angry player, $j_2$ is a convergence turn, thus $Loss_{A, j_2} = Loss_A$. We
       can also see that if $E$ had chosen the original evil strategy, the described actions would still be valid only by
       supplementing them with additional $Steal\left(\right)$ actions, thus $Loss_A$ would further increase. This proves the
       theorem.
    \end{sepproof}

    \begin{sepproof}{Proof of Lemma \ref{gameflow}: Transitive Games Are Flows} \ \\
    \label{gameflowproof}
       Let $Sad, Happy, Angry$ be as defined in the Transitive Game. Let $\mathcal{G}'$ be a directed weighted graph based on
       $\mathcal{G}$ with an auxiliary source. Let also $j_1$ be a turn when the Transitive Game has converged. More
       precisely, $\mathcal{G}'$ is defined as follows:
       \begin{equation*}
          \mathcal{V}' = \mathcal{V} \cup \{T\}
       \end{equation*}
       \begin{equation*}
          \mathcal{E}' = \mathcal{E} \cup \{(T, A)\} \cup \{(T, v) : v \in Sad_{j_1}\}
       \end{equation*}
       \begin{equation*}
          \forall (v, w) \in \mathcal{E}, c'_{vw} = DTr_{v \rightarrow w, 0} - DTr_{v \rightarrow w, j_1}
       \end{equation*}
       \begin{equation*}
          \forall v \in Sad_{j_1}, c'_{Tv} = c'_{TA} = \infty
       \end{equation*}
       \begin{center}
\begin{tikzpicture}[>=latex,line join=bevel,]
%%
\begin{scope}
  \definecolor{strokecol}{rgb}{0.0,0.0,0.0};
  \pgfsetstrokecolor{strokecol}
\end{scope}
\begin{scope}
  \pgfsetstrokecolor{black}
  \definecolor{strokecol}{rgb}{1.0,1.0,1.0};
  \pgfsetstrokecolor{strokecol}
  \definecolor{fillcol}{rgb}{1.0,1.0,1.0};
  \pgfsetfillcolor{fillcol}
  \filldraw (0.0bp,0.0bp) -- (0.0bp,168.0bp) -- (386.0bp,168.0bp) -- (386.0bp,0.0bp) -- cycle;
  \definecolor{strokecol}{rgb}{0.0,0.0,0.0};
  \pgfsetstrokecolor{strokecol}
  \draw (171.0bp,11.5bp) node {\textbf{Fig.\figlabel{fig:trgareflows}:} Graph $\mathcal{G}'$, derived from $\mathcal{G}$ with Auxiliary Source $T$.};
\end{scope}
\begin{scope}
  \pgfsetstrokecolor{black}
  \definecolor{strokecol}{rgb}{0.0,0.0,0.0};
  \pgfsetstrokecolor{strokecol}
  \draw (140.0bp,31.0bp) .. controls (140.0bp,31.0bp) and (276.0bp,31.0bp)  .. (276.0bp,31.0bp) .. controls (282.0bp,31.0bp) and (288.0bp,37.0bp)  .. (288.0bp,43.0bp) .. controls (288.0bp,43.0bp) and (288.0bp,148.0bp)  .. (288.0bp,148.0bp) .. controls (288.0bp,154.0bp) and (282.0bp,160.0bp)  .. (276.0bp,160.0bp) .. controls (276.0bp,160.0bp) and (140.0bp,160.0bp)  .. (140.0bp,160.0bp) .. controls (134.0bp,160.0bp) and (128.0bp,154.0bp)  .. (128.0bp,148.0bp) .. controls (128.0bp,148.0bp) and (128.0bp,43.0bp)  .. (128.0bp,43.0bp) .. controls (128.0bp,37.0bp) and (134.0bp,31.0bp)  .. (140.0bp,31.0bp);
  \draw (208.0bp,42.5bp) node {$\mathcal{G}$};
\end{scope}
  \node (A) at (163.0bp,80.0bp) [draw,ellipse] {A};
  \node (S) at (163.0bp,134.0bp) [draw,ellipse] {$\mathcal{S}$};
  \node (T) at (73.0bp,107.0bp) [draw,ellipse] {T};
  \node (G) at (233.0bp,107.0bp) [draw,ellipse] {$\mathcal{G} \setminus \left(\mathcal{S} \cup \{A\}\right)$};
  \draw [->] (A) ..controls (198.61bp,90.619bp) and (211.12bp,94.457bp)  .. (G);
  \draw [->] (S) ..controls (198.61bp,123.38bp) and (211.12bp,119.54bp)  .. (G);
  \draw [->] (T) ..controls (108.61bp,117.62bp) and (121.12bp,121.46bp)  .. (S);
  \draw (116.0bp,125.5bp) node {$\infty$};
  \draw [->] (T) ..controls (108.61bp,96.381bp) and (121.12bp,92.543bp)  .. (A);
  \draw (116.0bp,99.5bp) node {$\infty$};
%
\end{tikzpicture}
       \end{center}
       In the figure above, $\mathcal{S}$ is the set of sad players. We observe that $\forall v \in \mathcal{V},$
       \begin{equation}
       \label{gameflowin}
       \begin{gathered}
          \sum\limits_{w \in N^{-}\left(v\right)' \setminus \{T\}}c'_{wv} = \\
          = \sum\limits_{w \in N^{-}\left(v\right)' \setminus \{T\}}\left(DTr_{w \rightarrow v, 0} -
          DTr_{w \rightarrow v, j_1}\right) = \\
          = \sum\limits_{w \in N^{-}\left(v\right)' \setminus \{T\}}DTr_{w \rightarrow v, 0} -
          \sum\limits_{w \in N^{-}\left(v\right)' \setminus \{T\}}DTr_{w \rightarrow v, j-1} =  \\
          = in_{v, 0} - in_{v, j_1}
       \end{gathered}
       \end{equation}
       and
       \begin{equation}
       \label{gameflowout}
       \begin{gathered}
          \sum\limits_{w \in N^{+}\left(v\right)' \setminus \{T\}}c'_{vw} = \\
          = \sum\limits_{w \in N^{+}\left(v\right)' \setminus \{T\}}\left(DTr_{v \rightarrow w, 0} -
          DTr_{v \rightarrow w, j_1}\right) = \\
          = \sum\limits_{w \in N^{+}\left(v\right)' \setminus \{T\}}DTr_{v \rightarrow w, 0} -
          \sum\limits_{w \in N^{+}\left(v\right)' \setminus \{T\}}DTr_{v \rightarrow w, j-1} = \\
          = out_{v, 0} - out_{v, j_1} \enspace.
       \end{gathered}
       \end{equation}
       We can suppose that
       \begin{equation}
       \label{Aincoming}
          \forall j \in \mathbb{N}, in_{A, j} = 0 \enspace,
       \end{equation}
       since if we find a valid flow under this assumption, the flow will still be valid for the original graph. \\
       Next we try to calculate $MaxFlow\left(T, E\right) = X'$ on graph $\mathcal{G}'$. We observe that a flow in which it
       holds that $\forall v, w \in \mathcal{V}, x'_{vw} = c'_{vw}$ can be valid for the following reasons:
       \begin{itemize}
          \item $\forall v,w \in \mathcal{V}, x'_{vw} \leq c'_{vw}$ (Capacity flow requirement (\ref{flow1}) $\forall e \in
          \mathcal{E}$)
          \item Since $\forall v \in Sad_{j_1} \cup \{A\}, c'_{Tv} = \infty$, requirement (\ref{flow1}) holds for any flow
          $x'_{Tv} \geq 0$.
          \item Let $v \in \mathcal{V}' \setminus \left(Sad_{j_1} \cup \{T, A, E\}\right)$. According to the conservative
          strategy and since $v \notin Sad_{j_1},$ it holds that
          \begin{equation*}
             out_{v, 0} - out_{v, j_1} = in_{v, 0} - in_{v, j_1} \enspace.
          \end{equation*}
          Combining this observation with (\ref{gameflowin}) and (\ref{gameflowout}), we have that
          \begin{equation*}
             \sum\limits_{w \in \mathcal{V}'}c'_{vw} = \sum\limits_{w \in \mathcal{V}'}c'_{wv} \enspace.
          \end{equation*}
          (Flow Conservation requirement (\ref{flow2}) $\forall v \in \mathcal{V}' \setminus \left(Sad_{j_1}
          \cup \{T, A, E\}\right)$)
          \item Let $v \in Sad_{j_1}$. Since $v$ is sad, we know that
          \begin{equation*}
             out_{v, 0} - out_{v, j_1} > in_{v, 0} - in_{v, j_1} \enspace.
          \end{equation*}
          Since $c'_{Tv} = \infty$, we can set
          \begin{equation*}
             x'_{Tv} = \left(out_{v, 0} - out_{v, j_1}\right) - \left(in_{v, 0} - in_{v, j_1}\right) \enspace.
          \end{equation*}
          In this way, we have
          \begin{equation*}
             \sum\limits_{w \in \mathcal{V}'}x'_{vw} = out_{v, 0} - out_{v, j_1} \mbox{ and}
          \end{equation*}
          \begin{equation*}
          \begin{gathered}
             \sum\limits_{w \in \mathcal{V}'}x'_{wv} = \sum\limits_{w \in \mathcal{V}' \setminus \{T\}}c'_{wv} + x'_{Tv} =
             in_{v, 0} - in_{v, j_1} + \\ + (out_{v, 0} - out_{v, j_1}) - (in_{v, 0} - in_{v, j_1}) = out_{v, 0} -
             out_{v, j_1} \enspace.
          \end{gathered}
          \end{equation*}
          thus
          \begin{equation*}
             \sum\limits_{w \in \mathcal{V}'}x'_{vw} = \sum\limits_{w \in \mathcal{V}'}x'_{wv} \enspace.
          \end{equation*}
          (Requirement \ref{flow2} $\forall v \in Sad_{j_1}$)
          \item Since $c'_{TA} = \infty$, we can set
          \begin{equation*}
             x'_{TA} = \sum\limits_{v \in \mathcal{V}'}x'_{Av} \enspace,
          \end{equation*}
          thus from (\ref{Aincoming}) we have
          \begin{equation*}
             \sum\limits_{v \in \mathcal{V}'}x'_{vA} = \sum\limits_{v \in \mathcal{V}'}x'_{Av} \enspace.
          \end{equation*}
          (Requirement \ref{flow2} for $A$)
       \end{itemize}
       We saw that for all nodes, the necessary properties for a flow to be valid hold and thus $X'$ is a valid flow for
       $\mathcal{G}$. Moreover, this flow is equal to $maxFlow(T, E)$ because all incoming flows to $E$ are saturated.
       Also we observe that
       \begin{equation}
       \label{xprimeequalloss}
          \sum\limits_{v \in \mathcal{V}'}x'_{Av} = \sum\limits_{v \in \mathcal{V}'}c'_{Av} = out_{A, 0} - out_{A, j_1} =
          Loss_A \enspace.
       \end{equation}
       We define another graph, $\mathcal{G}''$, based on $\mathcal{G}'$.
       \begin{equation*}
          \mathcal{V}'' = \mathcal{V}'
       \end{equation*}
       \begin{equation*}
          E(\mathcal{G}'') = E(\mathcal{G}') \setminus \{(T, v) : v \in Sad_j\}
       \end{equation*}
       \begin{equation*}
          \forall e \in E(\mathcal{G}''), c''_e = c'_e
       \end{equation*}
       If we execute $MaxFlow(T, E)$ on the graph $\mathcal{G}''$, we will obtain a flow $X''$ in which
       \begin{equation*}
          \sum\limits_{v \in \mathcal{V}''}x''_{Tv} = x''_{TA} = \sum\limits_{v \in \mathcal{V}''}x''_{Av} \enspace.
       \end{equation*}
       The outgoing flow from $A$ in $X''$ will remain the same as in $X'$ for two reasons: Firstly, using the Flow
       Decomposition theorem \cite{amo} and deleting the paths that contain edges $\left(T, v\right): v \neq A$, we
       obtain a flow configuration where the total outgoing flow from $A$ remains invariant,
%       \footnote{We thank Kyriakos Axiotis for his insights on the Flow Decomposition theorem.}
       thus
       \begin{equation*}
          \sum\limits_{v \in \mathcal{V}''}x''_{Av} \geq \sum\limits_{v \in \mathcal{V}'}x'_{Av} \enspace.
       \end{equation*}
       Secondly, we have
       \begin{equation*}
          \begin{rcases}
             \sum\limits_{v \in \mathcal{V}''}c''_{Av} = \sum\limits_{v \in \mathcal{V}'}c'_{Av} = \sum\limits_{v \in
             \mathcal{V}'}x'_{Av} \\
             \sum\limits_{v \in \mathcal{V}''}c''_{Av} \geq \sum\limits_{v \in \mathcal{V}''}x''_{Av}
          \end{rcases}
          \Rightarrow \sum\limits_{v \in \mathcal{V}''}x''_{Av} \leq \sum\limits_{v \in \mathcal{V}'}x'_{Av} \enspace.
       \end{equation*}
       Thus we conclude that
       \begin{equation}
       \label{primeequaldoubleprime}
          \sum\limits_{v \in \mathcal{V}''}x''_{Av} = \sum\limits_{v \in \mathcal{V}'}x'_{Av} \enspace.
       \end{equation}
       Let $X = X'' \setminus \{(T, A)\}$. Observe that
       \begin{equation*}
          \sum\limits_{v \in \mathcal{V}''}x''_{Av} = \sum\limits_{v \in \mathcal{V}}x_{Av} \enspace.
       \end{equation*}
       This flow is valid on graph $\mathcal{G}$ because
       \begin{equation*}
          \forall e \in \mathcal{E}, c_e \geq c''_e \enspace.
       \end{equation*}
       Thus there exists a valid flow for each execution of the Transitive Game such that
       \begin{equation*}
          \sum\limits_{v \in \mathcal{V}}x_{Av} = \sum\limits_{v \in \mathcal{V}''}x''_{Av}
          \overset{\left(\ref{primeequaldoubleprime}\right)}{=} \sum\limits_{v \in \mathcal{V}'}x'_{Av}
          \overset{\left(\ref{xprimeequalloss}\right)}{=} Loss_{A, j_1} \enspace,
       \end{equation*}
       which is the flow $X$.
    \end{sepproof}

    \begin{theorem}[Conservative World Theorem] \ \\
       \label{conservativeworld}
       If everybody follows the conservative strategy, nobody steals any amount from anybody.
    \end{theorem}
     \begin{proof}
        Let $\mathcal{H}$ be the game history where all players are conservative and suppose there are some
        $Steal\left(\right)$ actions taking place. Then let $\mathcal{H}'$ be the subsequence of turns each containing at
        least one $Steal\left(\right)$ action. This subsequence is evidently nonempty, thus it must have a first element. The
        player corresponding to that turn, $A$, has chosen a $Steal\left(\right)$ action and no previous player has chosen
        such an action. However, player $A$ follows the conservative strategy, which is a contradiction.
     \end{proof}

    \begin{sepproof}{Proof of Theorem \ref{sybil}: Sybil Resilience} \ \\
    \label{sybilproof}
       Let $\mathcal{G}_1$ be a game graph defined as follows:
       \begin{equation*}
          \mathcal{V}_1 = \mathcal{V} \cup \{T_1\} \enspace,
       \end{equation*}
       \begin{equation*}
          \mathcal{E}_1 = \mathcal{E} \cup \{(v, T_1) : v \in \mathcal{B} \cup \mathcal{C}\} \enspace,
       \end{equation*}
       \begin{equation*}
          \forall v,w \in \mathcal{V}_1 \setminus \{T_1\}, DTr^1_{v \rightarrow w} = DTr_{v \rightarrow w} \enspace,
       \end{equation*}
       \begin{equation*}
          \forall v \in \mathcal{B} \cup \mathcal{C}, DTr^1_{v \rightarrow T_1} = \infty \enspace,
       \end{equation*}
       where $DTr_{v \rightarrow w}$ is the direct trust from $v$ to $w$ in $\mathcal{G}$ and $DTr^1_{v \rightarrow w}$ is
       the direct trust from $v$ to $w$ in $\mathcal{G}_1$. \\
       Let also $\mathcal{G}_2$ be the induced graph that results from $\mathcal{G}_1$ if we remove the Sybil set,
       $\mathcal{C}$. We rename $T_1$ to $T_2$ and define $\mathcal{L} = \mathcal{V} \setminus \left(\mathcal{B} \cup
       \mathcal{C}\right)$ as the set of legitimate players to facilitate comprehension.
       \begin{center}
\begin{tikzpicture}[>=latex,line join=bevel,scale=0.7,transform shape]
%%
\begin{scope}
  \definecolor{strokecol}{rgb}{0.0,0.0,0.0};
  \pgfsetstrokecolor{strokecol}
\end{scope}
\begin{scope}
  \pgfsetstrokecolor{black}
  \definecolor{strokecol}{rgb}{1.0,1.0,1.0};
  \pgfsetstrokecolor{strokecol}
  \definecolor{fillcol}{rgb}{1.0,1.0,1.0};
  \pgfsetfillcolor{fillcol}
  \filldraw (0.0bp,0.0bp) -- (0.0bp,186.0bp) -- (467.0bp,186.0bp) -- (467.0bp,0.0bp) -- cycle;
  \definecolor{strokecol}{rgb}{0.0,0.0,0.0};
  \pgfsetstrokecolor{strokecol}
  \draw (233.5bp,11.5bp) node {\LARGE \textbf{Fig.\figlabel{fig:sybilres}:} Graphs $\mathcal{G}_1$ and $\mathcal{G}_2$};
\end{scope}
\begin{scope}
  \pgfsetstrokecolor{black}
  \definecolor{strokecol}{rgb}{0.0,0.0,0.0};
  \pgfsetstrokecolor{strokecol}
  \draw (20.0bp,31.0bp) .. controls (20.0bp,31.0bp) and (121.0bp,31.0bp)  .. (121.0bp,31.0bp) .. controls (127.0bp,31.0bp) and (133.0bp,37.0bp)  .. (133.0bp,43.0bp) .. controls (133.0bp,43.0bp) and (133.0bp,166.0bp)  .. (133.0bp,166.0bp) .. controls (133.0bp,172.0bp) and (127.0bp,178.0bp)  .. (121.0bp,178.0bp) .. controls (121.0bp,178.0bp) and (20.0bp,178.0bp)  .. (20.0bp,178.0bp) .. controls (14.0bp,178.0bp) and (8.0bp,172.0bp)  .. (8.0bp,166.0bp) .. controls (8.0bp,166.0bp) and (8.0bp,43.0bp)  .. (8.0bp,43.0bp) .. controls (8.0bp,37.0bp) and (14.0bp,31.0bp)  .. (20.0bp,31.0bp);
  \draw (70.5bp,42.5bp) node {\LARGE $\mathcal{G}_1$};
\end{scope}
\begin{scope}
  \pgfsetstrokecolor{black}
  \definecolor{strokecol}{rgb}{0.0,0.0,0.0};
  \pgfsetstrokecolor{strokecol}
  \draw (20.0bp,31.0bp) .. controls (20.0bp,31.0bp) and (121.0bp,31.0bp)  .. (121.0bp,31.0bp) .. controls (127.0bp,31.0bp) and (133.0bp,37.0bp)  .. (133.0bp,43.0bp) .. controls (133.0bp,43.0bp) and (133.0bp,166.0bp)  .. (133.0bp,166.0bp) .. controls (133.0bp,172.0bp) and (127.0bp,178.0bp)  .. (121.0bp,178.0bp) .. controls (121.0bp,178.0bp) and (20.0bp,178.0bp)  .. (20.0bp,178.0bp) .. controls (14.0bp,178.0bp) and (8.0bp,172.0bp)  .. (8.0bp,166.0bp) .. controls (8.0bp,166.0bp) and (8.0bp,43.0bp)  .. (8.0bp,43.0bp) .. controls (8.0bp,37.0bp) and (14.0bp,31.0bp)  .. (20.0bp,31.0bp);
  \draw (70.5bp,42.5bp) node {\LARGE $\mathcal{G}_1$};
\end{scope}
\begin{scope}
  \pgfsetstrokecolor{black}
  \definecolor{strokecol}{rgb}{0.0,0.0,0.0};
  \pgfsetstrokecolor{strokecol}
  \draw (264.0bp,67.0bp) .. controls (264.0bp,67.0bp) and (365.0bp,67.0bp)  .. (365.0bp,67.0bp) .. controls (371.0bp,67.0bp) and (377.0bp,73.0bp)  .. (377.0bp,79.0bp) .. controls (377.0bp,79.0bp) and (377.0bp,130.0bp)  .. (377.0bp,130.0bp) .. controls (377.0bp,136.0bp) and (371.0bp,142.0bp)  .. (365.0bp,142.0bp) .. controls (365.0bp,142.0bp) and (264.0bp,142.0bp)  .. (264.0bp,142.0bp) .. controls (258.0bp,142.0bp) and (252.0bp,136.0bp)  .. (252.0bp,130.0bp) .. controls (252.0bp,130.0bp) and (252.0bp,79.0bp)  .. (252.0bp,79.0bp) .. controls (252.0bp,73.0bp) and (258.0bp,67.0bp)  .. (264.0bp,67.0bp);
  \draw (314.5bp,78.5bp) node {\LARGE $\mathcal{G}_2$};
\end{scope}
  \node (T2) at (449.0bp,116.0bp) [draw,ellipse] {\LARGE $T_2$};
  \node (T1) at (205.0bp,84.0bp) [draw,ellipse] {\LARGE $T_1$};
  \node (V1) at (34.0bp,80.0bp) [draw,ellipse] {\LARGE $\mathcal{L}$};
  \node (V2) at (278.0bp,116.0bp) [draw,ellipse] {\LARGE $\mathcal{L}$};
  \node (B1) at (107.0bp,80.0bp) [draw,ellipse] {\LARGE $\mathcal{B}$};
  \node (B2) at (351.0bp,116.0bp) [draw,ellipse] {\LARGE $\mathcal{B}$};
  \node (C1) at (107.0bp,152.0bp) [draw,ellipse] {\LARGE $\mathcal{C}$};
  \draw [->] (C1) ..controls (138.76bp,130.21bp) and (166.32bp,110.69bp)  .. (T1);
  \draw (156.0bp,133.5bp) node {\LARGE $\infty$};
  \draw [->] (C1) ..controls (92.873bp,124.6bp) and (92.196bp,113.02bp)  .. (B1);
  \draw [->] (V2) ..controls (304.6bp,109.16bp) and (316.75bp,108.964bp)  .. (B2);
  \draw [->] (B1) ..controls (121.11bp,107.33bp) and (121.81bp,118.91bp)  .. (C1);
  \draw [->] (B2) ..controls (384.73bp,116.0bp) and (407.79bp,116.0bp)  .. (T2);
  \draw (400.0bp,123.5bp) node {\LARGE $\infty$};
  \draw [->] (B1) ..controls (140.73bp,81.364bp) and (163.79bp,82.325bp)  .. (T1);
  \draw (156.0bp,90.5bp) node {\LARGE $\infty$};
  \draw [->] (V1) ..controls (60.598bp,73.16bp) and (72.75bp,72.964bp)  .. (B1);
  \draw [->] (B2) ..controls (323.89bp,122.861bp) and (311.73bp,123.031bp)  .. (V2);
  \draw [->] (C1) ..controls (82.216bp,127.88bp) and (64.772bp,110.19bp)  .. (V1);
  \draw [->] (B1) ..controls (79.894bp,86.861bp) and (67.729bp,87.031bp)  .. (V1);
%
\end{tikzpicture}
       \end{center}
       According to theorem (\ref{trustmany}),
       \begin{equation}
       \label{trmaxflow}
          Tr_{A \rightarrow \mathcal{B} \cup \mathcal{C}} = maxFlow_1\left(A, T_1\right) \wedge
          Tr_{A \rightarrow \mathcal{B}} = maxFlow_2\left(A, T_2\right) \enspace.
       \end{equation}
       We will show that the $MaxFlow$ of each of the two graphs can be used to construct a valid flow of equal value for the
       other graph. The flow $X_1 = MaxFlow\left(A, T_1\right)$ can be used to construct a valid flow of equal value for the
       second graph if we set
       \begin{align*}
          \forall v \in \mathcal{V}_2 \setminus \mathcal{B}, \forall w \in \mathcal{V}_2&, x_{vw,2} = x_{vw,1} \enspace, \\
          \forall v \in \mathcal{B}&, x_{vT_2,2} = \sum\limits_{w \in N^{+}_1\left(v\right)}x_{vw,1} \enspace, \\
          \forall v,w \in \mathcal{B}&, x_{vw,2} = 0 \enspace.
       \end{align*}
       Therefore
       \begin{equation*}
          maxFlow_1\left(A, T_1\right) \leq maxFlow_2\left(A, T_2\right)
       \end{equation*}
       Likewise, the flow $X_2 = MaxFlow(A, T_2)$ is a valid flow for $\mathcal{G}_1$ because $\mathcal{G}_2$ is an induced
       subgraph of $\mathcal{G}_1$. Therefore
       \begin{equation*}
          maxFlow_1\left(A, T_1\right) \geq maxFlow_2\left(A, T_2\right)
       \end{equation*}
       We conclude that
       \begin{equation}
       \label{eqmaxflows}
          maxFlow\left(A, T_1\right) = maxFlow\left(A, T_2\right) \enspace,
       \end{equation}
       thus from (\ref{trmaxflow}) and (\ref{eqmaxflows}) the theorem holds.
    \end{sepproof}

