\Suppressnumber
\begin{lstlisting}[label=abs, style=numbers]
Absolute Equality Trust Transfer ( (*@$||\Delta_i||_{\infty}$@*) minimizer)
Input : old flows (*@$x_i$@*), value V
Output : new capacities (*@$c'_i$ \Reactivatenumber@*)
abs((*@$\left(x_i\right)$@*), V) :
  n = length((*@$x_i$@*))
  F(*@$_{cur}$@*) = F = (*@$\sum\limits_{i=1}^nx_i$ \label{abs:finit}@*)
  if (F < V) (*@\label{abs:ifinputvalid}@*)
    return((*@$\bot$@*)) (*@\label{abs:inputinvalid}@*)
  X = preprocess((*@$x_i$@*)) (*@\label{abs:preprocess}@*)
  empty = 0 (*@\label{abs:emptyinit}@*)
  reduction = 0 (*@\label{abs:reductioninit}@*)
  while (F(*@$_{cur}$@*) > F - V) (*@\label{abs:loop}@*)
    ((*@$i$@*), X) = popMin(X) (*@\label{abs:popmin}@*)
    F(*@$_{prov}$@*) = F(*@$_{cur}$@*) - (n - empty)*((*@$x_i$@*) - reduction) (*@\label{abs:setfprov}@*)
    if (F(*@$_{prov}$@*) > F - V) (*@\label{abs:iffprovbig}@*)
      reduction = (*@$x_i$@*) (*@\label{abs:setreduction}@*)
      empty += 1 (*@\label{abs:setempty}@*)
      F(*@$_{cur}$@*) = F(*@$_{prov}$@*) (*@\label{abs:setfcur}@*)
    else (*@\label{abs:elsefprovsmall}@*)
      aux = reduction (*@\label{abs:saveprevreduction}@*)
      reduction += (*@$\frac{\mbox{F}_{cur}\mbox{ - (F - V)}}{\mbox{n - empty}}$@*) (*@\label{abs:setfinalreduction}@*)
      F(*@$_{cur}$@*) -= (n - empty)*(reduction - aux) (*@\label{abs:setlastfcur}@*)
      #lines(*@~\ref{abs:saveprevreduction}@*) &(*@~\ref{abs:setlastfcur}@*) can be replaced by break. In this
      #case, the loop (line(*@~\ref{abs:loop}@*)) can become while (TRUE).
  for ((*@$i$@*) = 1 to n) (*@\label{abs:forsetcap}@*)
    (*@$c'_i$@*) = max(0, (*@$x_i$@*) - reduction) (*@\label{abs:setcap}@*)
  return((*@$\bigcup\limits_{i=1}^n\{c'_i\}$@*)) (*@\label{abs:return}@*)
\end{lstlisting}

The function \texttt{preprocess(}$x_i$\texttt{)} returns a data structure \texttt{X} containing the set of flows
$\left(x_i\right)$, such that the corresponding function \texttt{popMin(X)} is able to repeatedly return the index of a tuple
consisiting of the index of the minimum element and a new data structure missing exactly the minimum element. Examples of
such pairs of functions are:
\begin{equation*}
\begin{gathered}
  \begin{cases}
    \texttt{preprocess = quickSort} \\
    \texttt{popMin = (}x_1\texttt{, X}\setminus x_1\texttt{)}
  \end{cases}
  \mbox{ and} \\
  \begin{cases}
    \texttt{preprocess = FibonacciHeap} \\
    \texttt{popMin = (find-min(X),delete-min(X))}
  \end{cases} \enspace.
\end{gathered}
\end{equation*}
