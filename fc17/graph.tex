\section{The Trust Graph}
  We now engage in the formal description of the proposed system, accompanied by helpful examples.
  \subimport{common/definitions/}{gamegraph.tex}
  \noindent The nodes represent the players, the edges represent the existing direct trusts and the weights represent the
  amount of value attached to the corresponding direct trust. As we will see, the game evolves in turns. The subscript of the
  graph represents the corresponding turn.
  \subimport{common/definitions/}{players.tex}
  \noindent Each node has a corresponding non-negative number that represents its capital. A node's capital is the total value
  that the node possesses exclusively and nobody else can spend.
  \subimport{common/definitions/}{capital.tex}
  \noindent The capital is the value that exists in the game but is not shared with trusted parties. The capital of $A$ can be
  reallocated only during her turns, according to her actions. We model the system in a way that no capital can be added in
  the course of the game through external means. The use of capital will become clear once turns are formally defined.

  The formal definition of direct trust follows:
  \subimport{common/definitions/}{directtrust.tex}
  \subimport{common/figures/}{utxo.tikz}
  \noindent The definition of direct trust agrees with the title of this paper and coincides with the intuition and sociological experimental
  results of Karlan et al. \cite{kmrs} that the trust $Alice$ shows to $Bob$ in real-world social networks corresponds to the
  extent of danger in which $Alice$ is putting herself into in order to help $Bob$. An example graph with its corresponding
  transactions in the UTXO can be seen in Fig.~\ref{fig:utxo}.


  Any algorithm that has access to the graph $\mathcal{G}_j$ has implicitly access to all direct trusts of this graph.

  \subimport{common/definitions/}{neighbourhood.tex}
  \subimport{common/definitions/}{inouttrust.tex}
  \subimport{common/definitions/}{assets.tex}
