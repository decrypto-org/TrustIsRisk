  \section{Trust Transitivity}
     In this section we define some strategies. The corresponding algorithms can be seen in the Appendix. Then we define the
     Transitive Game that represents the worst-case scenario for an honest player when another player decides to depart from
     the network with her money and all the money entrusted to her.
     \begin{definition}[Idle Strategy]
        A player $A$ is said to follow the idle strategy if she passes in her turn. 
     \end{definition}

     \Suppressnumber
     \begin{lstlisting}[label=idlestrategy, style=numbers]
Idle Strategy
Input : initial graph (*@$\mathcal{G}_0$@*), player (*@$A$@*), history (*@$\left(\mathcal{H}\right)_{1 \dots j-1}$@*)
Output : (*@$Turn_j$ \Reactivatenumber@*)
idleStrategy((*@$\mathcal{G}_0$@*), (*@$A$@*), (*@$\mathcal{H}$@*)) :
  return((*@$\emptyset$@*))
     \end{lstlisting}
     The inputs and outputs are identical to those of \texttt{idleStrategy()} for the rest of the strategies, thus we avoid
     repeating them.
     \begin{definition}[Evil Strategy]
        A player $A$ is said to follow the evil strategy if she steals all incoming direct trust and nullifies her outgoing
        direct trust in her turn.
     \end{definition}

     \begin{lstlisting}[label=evilstrategy, style=numbers]
evilStrategy((*@$\mathcal{G}_0$@*), (*@$A$@*), (*@$\mathcal{H}$@*)) :
  Steals = (*@$\bigcup\limits_{v \in N^{-}\left(A\right)_{j-1}}\{Steal(DTr_{v \rightarrow A, j-1}, v)\}$@*)
  Adds = (*@$\bigcup\limits_{v \in N^{+}\left(A\right)_{j-1}}\{Add(-DTr_{A \rightarrow v, j-1}, v)\}$@*)
  (*@$Turn_j$@*) = Steals(*@$\: \cup \:$@*)Adds
  return((*@$Turn_j$@*))
     \end{lstlisting}

     \begin{definition}[Conservative Strategy]
        Player $A$ is said to follow the conservative strategy if she replenishes the value she lost since the previous turn,
        $Damage_A$, by stealing from others that trust her as much as she can up to $Damage_A$ and she takes no other action.
     \end{definition}

     \begin{lstlisting}[label=conservativestrategy, style=numbers]
consStrategy((*@$\mathcal{G}_0$@*), (*@$A$@*), (*@$\mathcal{H}$@*)) :
  Damage = (*@$out_{A, prev\left(j\right)}$@*) - (*@$out_{A, j-1}$@*)
  if (Damage > 0)
    if (Damage >= (*@$in_{A, j-1}$@*))
      (*@$Turn_j$@*) = (*@$\bigcup\limits_{v \in N^{-}\left(A\right)_{j-1}}\{Steal\left(DTr_{v \rightarrow A, j-1}, v\right)\}$@*)
    else
      (*@$y$@*) = SelectSteal((*@$G_j$@*), (*@$A$@*), Damage)    #(*@$y$@*) = (*@$\{y_v : v \in N^{-}\left(A\right)_{j-1}\}$@*)
      (*@$Turn_j$@*) = (*@$\bigcup\limits_{v \in N^{-}\left(A\right)_{j-1}}\{Steal\left(y_v, v\right)\}$@*)
  else (*@$Turn_j$@*) = (*@$\emptyset$@*)
  return((*@$Turn_j$@*))
     \end{lstlisting}
     \texttt{SelectSteal()} returns $y_v$ with $v \in N^{-}\left(A\right)_{j-1}$ such that
     \begin{equation}
     \label{stealrestriction}
        \sum\limits_{v \in N^{-}\left(A\right)_{j-1}}y_v = Damage_{A, j} \wedge \forall v \in N^{-}\left(A\right)_{j-1},
        y_v \leq DTr_{v \rightarrow A, j-1} \enspace.
     \end{equation}
     Player $A$ can arbitrarily define how \texttt{SelectSteal()} distributes the $Steal\left(\right)$ actions
     each time she calls the function, as long as (\ref{stealrestriction}) is respected. 

     As we can see, the definition covers a multitude of options for the conservative player, since in case $0 < Damage_{A,j}
     < in_{A,j-1}$ she can choose to distribute the $Steal\left(\right)$ actions in any way she chooses.

     The rationale behind this strategy arises from a real-world common situation. Suppose there are a client, an
     intermediary and a producer. The client entrusts some value to the intermediary so that the latter can buy the desired
     product from the producer and deliver it to the client. The intermediary in turn entrusts an equal value to the
     producer, who needs the value upfront to be able to complete the production process. However the producer eventually
     does not give the product neither reimburses the value, due to bankruptcy or decision to exit the market with an unfair
     benefit. The intermediary can choose either to reimburse the client and suffer the loss, or refuse to return the money
     and lose the client's trust. The latter choice for the intermediary is exactly the conservative strategy. It is used
     throughout this work as a strategy for all the intermediary players because it models effectively the worst-case
     scenario that a client can face after an evil player decides to steal everything she can and the rest of the players do
     not engage in evil activity.

     We continue with a very useful possible evolution of the game, the Transitive Game. In turn 0, there is already a network
     in place. All players apart from $A$ and $E$ follow the conservative strategy. Furthermore, the set of players is not
     modified throughout the Transitive Game, thus we can refer to $\mathcal{V}_j$ for any turn $j$ as $\mathcal{V}$.
     Moreover, each conservative player can be in one of three states: Happy, Angry or Sad. Happy players have 0 loss, Angry
     players have positive loss and positive incoming trust, thus are able to replenish their loss at least in part and
     Sad players have positive loss, but 0 incoming trust, thus they cannot replenish the loss. These conventions will hold
     whenever we use the Transitive Game.
% For the algorithm, see the Appendix.
     \Suppressnumber
     \begin{lstlisting}[label=transitivegame, style=numbers]
Transitive Game
Input : graph (*@$\mathcal{G}_0$@*), (*@$A \in \mathcal{V}$@*) idle player, (*@$E \in \mathcal{V}$@*) evil player (*@\Reactivatenumber@*)
Angry = Sad = (*@$\emptyset$@*) (*@\label{trstealsadinit}@*);  Happy = (*@$\mathcal{V} \setminus \{A, E\}$@*)
for ((*@$v \in \mathcal{V} \setminus \{E\}$@*))  (*@$Loss_v$@*) = 0 (*@\label{trsteallossinit}@*)
j = 0
while (True)
  j += 1;  (*@$v \overset{\$}{\gets} \mathcal{V} \setminus \{A\}$@*)
  (*@$Turn_j$@*) = (*@$v$@*)Strategy((*@$\mathcal{G}_0$@*), (*@$v$@*), (*@$\left(\mathcal{H}\right)_{1 \dots j-1}$@*))
  executeTurn((*@$\mathcal{G}_{j-1}$@*), (*@$Cap_{v, j-1}$@*), (*@$Turn_j$@*))
  for (action (*@$\in Turn_j$@*))
    action match do
      case (*@$Steal($@*)y(*@$,w)$@*) do
        exchange = y
        (*@$Loss_w$@*) += exchange (*@\label{trsteallossincrease}@*)
        if ((*@$v$@*) != (*@$E$@*)) (*@$Loss_v$@*) -= exchange (*@\label{trsteallossdecrease}@*)
        if ((*@$w$@*) != (*@$A$@*))
          Happy = Happy(*@$\:\setminus\: \{w\}$@*)
          if ((*@$in_{w, j}$@*) == 0) Sad = Sad(*@$\:\cup\: \{w\}$@*)
          else Angry = Angry(*@$\:\cup\: \{w\}$@*)
  if ((*@$v$@*) != (*@$E$@*))
    Angry = Angry(*@$\:\setminus\: \{v\}$@*)
    if ((*@$Loss_v$@*) > 0) (*@\label{trstealifentersad}@*)  Sad = Sad(*@$\:\cup\: \{v\}$@*)        #(*@$in_{v, j}$@*) should be zero (*@\label{trstealtrueentersad}@*)
    if ((*@$Loss_v$@*) == 0)  Happy = Happy(*@$\:\cup\: \{v\}$@*)
     \end{lstlisting}

     An example execution follows:

\begin{center}
\begin{tikzpicture}[>=latex,line join=bevel,scale=0.7,transform shape]
%%
\begin{scope}
  \definecolor{strokecol}{rgb}{0.0,0.0,0.0};
  \pgfsetstrokecolor{strokecol}
\end{scope}
\begin{scope}
  \pgfsetstrokecolor{black}
  \definecolor{strokecol}{rgb}{1.0,1.0,1.0};
  \pgfsetstrokecolor{strokecol}
  \definecolor{fillcol}{rgb}{1.0,1.0,1.0};
  \pgfsetfillcolor{fillcol}
  \filldraw (0.0bp,0.0bp) -- (0.0bp,320.0bp) -- (457.0bp,320.0bp) -- (457.0bp,0.0bp) -- cycle;
  \definecolor{strokecol}{rgb}{0.0,0.0,0.0};
  \pgfsetstrokecolor{strokecol}
\end{scope}
\begin{scope}
  \pgfsetstrokecolor{black}
  \definecolor{strokecol}{rgb}{1.0,1.0,1.0};
  \pgfsetstrokecolor{strokecol}
  \definecolor{fillcol}{rgb}{1.0,1.0,1.0};
  \pgfsetfillcolor{fillcol}
  \filldraw (0.0bp,0.0bp) -- (0.0bp,320.0bp) -- (457.0bp,320.0bp) -- (457.0bp,0.0bp) -- cycle;
  \definecolor{strokecol}{rgb}{0.0,0.0,0.0};
  \pgfsetstrokecolor{strokecol}
  \draw (228.5bp,11.5bp) node {\Large \textbf{Fig.\figlabel{fig:transitivegame}:} Turns of a \texttt{TransitiveGame(}$\mathcal{G}_0$\texttt{,}$A$\texttt{,}$E$\texttt{)}};
\end{scope}
\begin{scope}
  \pgfsetstrokecolor{black}
  \definecolor{strokecol}{rgb}{1.0,1.0,1.0};
  \pgfsetstrokecolor{strokecol}
  \draw (239.0bp,31.0bp) -- (239.0bp,179.0bp) -- (449.0bp,179.0bp) -- (449.0bp,31.0bp) -- cycle;
  \definecolor{strokecol}{rgb}{0.0,0.0,0.0};
  \pgfsetstrokecolor{strokecol}
  \draw (414.0bp,42.5bp) node {\Large $\mathcal{G}_3$};
\end{scope}
\begin{scope}
  \pgfsetstrokecolor{black}
  \definecolor{strokecol}{rgb}{1.0,1.0,1.0};
  \pgfsetstrokecolor{strokecol}
  \draw (8.0bp,33.0bp) -- (8.0bp,181.0bp) -- (218.0bp,181.0bp) -- (218.0bp,33.0bp) -- cycle;
  \definecolor{strokecol}{rgb}{0.0,0.0,0.0};
  \pgfsetstrokecolor{strokecol}
  \draw (183.0bp,44.5bp) node {\Large $\mathcal{G}_2$};
\end{scope}
\begin{scope}
  \pgfsetstrokecolor{black}
  \definecolor{strokecol}{rgb}{1.0,1.0,1.0};
  \pgfsetstrokecolor{strokecol}
  \draw (239.0bp,187.0bp) -- (239.0bp,320.0bp) -- (449.0bp,320.0bp) -- (449.0bp,187.0bp) -- cycle;
  \definecolor{strokecol}{rgb}{0.0,0.0,0.0};
  \pgfsetstrokecolor{strokecol}
  \draw (414.0bp,198.5bp) node {\Large $\mathcal{G}_1$};
\end{scope}
\begin{scope}
  \pgfsetstrokecolor{black}
  \definecolor{strokecol}{rgb}{1.0,1.0,1.0};
  \pgfsetstrokecolor{strokecol}
  \draw (8.0bp,202.0bp) -- (8.0bp,320.0bp) -- (218.0bp,320.0bp) -- (218.0bp,202.0bp) -- cycle;
  \definecolor{strokecol}{rgb}{0.0,0.0,0.0};
  \pgfsetstrokecolor{strokecol}
  \draw (183.0bp,198.5bp) node {\Large $\mathcal{G}_0$};
\end{scope}
  \node (E3) at (192.0bp,264.0bp) [draw,ellipse] {\Large E};
  \node (E2) at (423.0bp,264.0bp) [draw,ellipse,very thick] {\Large E};
  \node (D2) at (344.0bp,289.0bp) [draw,ellipse] {\Large D};
  \node (D2mood) at (344.0bp,310.0bp) {\Large Angry};
  \node (D3) at (113.0bp,289.0bp) [draw,ellipse] {\Large D};
  \node (D3mood) at (113.0bp,310.0bp) {\Large Happy};
  \node (A1) at (34.0bp,114.0bp) [draw,ellipse] {\Large A};
  \node (A0) at (265.0bp,114.0bp) [draw,ellipse] {\Large A};
  \node (A3) at (34.0bp,270.0bp) [draw,ellipse] {\Large A};
  \node (A2) at (265.0bp,270.0bp) [draw,ellipse] {\Large A};
  \node (B0) at (265.0bp,60.0bp) [draw,ellipse] {\Large B};
  \node (B0mood) at (265.0bp,39.0bp) {\Large Sad};
  \node (B1) at (34.0bp,60.0bp) [draw,ellipse] {\Large B};
  \node (B1mood) at (34.0bp,39.0bp) {\Large Happy};
  \node (B2) at (265.0bp,216.0bp) [draw,ellipse] {\Large B};
  \node (B2mood) at (265.0bp,195.0bp) {\Large Happy};
  \node (B3) at (34.0bp,216.0bp) [draw,ellipse] {\Large B};
  \node (B3mood) at (34.0bp,195.0bp) {\Large Happy};
  \node (C3) at (113.0bp,235.0bp) [draw,ellipse] {\Large C};
  \node (C3mood) at (113.0bp,214.0bp) {\Large Happy};
  \node (C2) at (344.0bp,235.0bp) [draw,ellipse] {\Large C};
  \node (C2mood) at (344.0bp,214.0bp) {\Large Angry};
  \node (C1) at (113.0bp,79.0bp) [draw,ellipse] {\Large C};
  \node (C1mood) at (113.0bp,58.0bp) {\Large Angry};
  \node (C0) at (344.0bp,79.0bp) [draw,ellipse,very thick] {\Large C};
  \node (C0mood) at (344.0bp,58.0bp) {\Large Happy};
  \node (E1) at (192.0bp,108.0bp) [draw,ellipse] {\Large E};
  \node (E0) at (423.0bp,108.0bp) [draw,ellipse] {\Large E};
  \node (D0) at (344.0bp,133.0bp) [draw,ellipse] {\Large D};
  \node (D0mood) at (344.0bp,154.0bp) {\Large Sad};
  \node (D1) at (113.0bp,133.0bp) [draw,ellipse,very thick] {\Large D};
  \node (D1mood) at (113.0bp,154.0bp) {\Large Sad};
  \draw [->] (B1) ..controls (59.732bp,58.525bp) and (68.942bp,58.948bp)  .. (77.0bp,61.0bp) .. controls (82.045bp,62.285bp) and (87.187bp,64.399bp)  .. (C1);
  \draw (73.5bp,68.5bp) node {\Large 7\bitcoin};
  \draw [->] (A3) ..controls (62.101bp,257.65bp) and (78.735bp,250.41bp)  .. (C3);
  \draw (73.5bp,261.5bp) node {\Large 6\bitcoin};
  \draw [->] (A1) ..controls (62.101bp,101.65bp) and (78.735bp,92.41bp)  .. (C1);
  \draw (73.5bp,105.5bp) node {\Large 6\bitcoin};
  \draw [->] (A2) ..controls (293.71bp,276.84bp) and (308.24bp,280.42bp)  .. (D2);
  \draw (304.5bp,288.5bp) node {\Large 3\bitcoin};
  \draw [->] (A0) ..controls (293.17bp,101.65bp) and (309.1bp,92.41bp)  .. (C0);
  \draw (304.5bp,105.5bp) node {\Large 4\bitcoin};
  \draw [->] (B3) ..controls (59.732bp,214.52bp) and (68.942bp,214.95bp)  .. (77.0bp,217.0bp) .. controls (82.045bp,218.28bp) and (87.187bp,220.4bp)  .. (C3);
  \draw (73.5bp,224.5bp) node {\Large 7\bitcoin};
  \draw [->] (B2) ..controls (290.73bp,214.52bp) and (299.94bp,214.95bp)  .. (308.0bp,217.0bp) .. controls (313.04bp,218.28bp) and (318.19bp,220.4bp)  .. (C2);
  \draw (304.5bp,224.5bp) node {\Large 7\bitcoin};
  \draw [->] (A2) ..controls (293.17bp,257.65bp) and (309.1bp,250.41bp)  .. (C2);
  \draw (304.5bp,261.5bp) node {\Large 6\bitcoin};
  \draw [->] (C3) ..controls (141.95bp,248.63bp) and (156.16bp,252.48bp)  .. (E3);
  \draw (152.5bp,260.5bp) node {\Large 3\bitcoin};
  \draw [->] (D3) ..controls (141.1bp,282.51bp) and (157.73bp,275.01bp)  .. (E3);
  \draw (152.5bp,286.5bp) node {\Large 4\bitcoin};
  \draw [->] (A3) ..controls (62.955bp,276.84bp) and (77.158bp,280.42bp)  .. (D3);
  \draw (73.5bp,288.5bp) node {\Large 3\bitcoin};
  \draw [->] (B0) ..controls (290.73bp,58.525bp) and (299.94bp,58.948bp)  .. (308.0bp,61.0bp) .. controls (313.04bp,62.285bp) and (318.19bp,64.399bp)  .. (C0);
  \draw (304.5bp,68.5bp) node {\Large 6\bitcoin};
  \draw (228.0bp,317.0bp) -- (228.0bp,35.0bp);
  \draw (0.0bp,169.0bp) -- (455.0bp,169.0bp);
%
\end{tikzpicture}
\end{center}

     Let $j_0$ be the first turn on which $E$ is chosen to play. Until then, all players will pass their turn since nothing
     has been stolen yet (see the Appendix (theorem \ref{conservativeworld}) for a formal proof of this simple fact).
     Moreover, let $v = Player(j)$ and $j' = prev\left(j\right)$.
%     Given that
%     \begin{equation}
%        Damage_{v,j} = out_{v, j'} - out_{v, j-1} \enspace,
%     \end{equation}
     The Transitive Game generates turns:
     \begin{equation}
        Turn_j = \bigcup\limits_{w \in N^{-}\left(v\right)_{j-1}}\{Steal\left(y_w,w\right)\} \enspace,% Damage_{v, j} > 0 \enspace,
     \end{equation}
     where
     \begin{equation*}
        \sum\limits_{w \in N^{-}\left(v\right)_{j-1}}y_w = \min\left(in_{v, j-1}, Damage_{v, j}\right) \enspace.
     \end{equation*}
 
     We see that if $Damage_{v, j} = 0$, then $Turn_j = \emptyset$.

     From the definition of $Damage_{v,j}$ and knowing that no strategy in this case can increase any direct trust, we see
     that $Damage_{v,j} \geq 0$. Also, it is $Loss_{v,j} \geq 0$ because if $Loss_{v,j} < 0$, then $v$ has
     stolen more value than she has been stolen, thus she would not be following the conservative strategy.
%    In the figure above, $D$ is angry after turn 1 and sad after turn 2 and on. $C$ is angry after turns 1 and 2 and happy
%    after turn 3. $B$ is happy until the end of turn 2 and sad after turn 3. The game converges in turn 4.
