\section{Sybil Resilience}
  One of our aims is to mitigate Sybil attacks \cite{sybilattack} whilst maintaining decentralized autonomy \cite{dionyziz}.
  We begin by extending the definition of indirect trust.
  \subimport{common/definitions/}{indirecttrustmultiplayer.tex}
  \noindent We now extend the Trust Flow theorem to many players.
  \subimport{common/theorems/}{multiplayertrustflowtheorem.tex}
  \subimport{common/proofs/}{multiplayertrustflowproof.tex}
  \noindent We now define several useful notions to tackle the problem of Sybil attacks. Let Eve be a possible attacker.
  \subimport{common/definitions/}{corrupted.tex}
  \subimport{common/definitions/}{sybil.tex}
  \subimport{common/definitions/}{collusion.tex}
  \subimport{common/figures/}{collusion.tikz}
  From a game theoretic point of view, players $\mathcal{V} \setminus (\mathcal{B} \cup \mathcal{C})$ perceive the collusion
  as independent players with a distinct strategy each, whereas in reality they are all subject to a single strategy dictated
  by Eve.
  \subimport{fc17/theorems/}{sybilrestheorem.tex}
  \subimport{fc17/proofsketches/}{sybilresproofsketch.tex}
  We have proven that controlling $|\mathcal{C}|$ is irrelevant for Eve, thus Sybil attacks are meaningless. Note that
  the theorem does not reassure against deception attacks. Specifically, a malicious player can create several identities, use
  them legitimately to inspire others to deposit direct trust to these identities and then switch to the evil strategy, thus
  defrauding everyone that trusted the fabricated identities. These identities correspond to the corrupted set of players and
  not to the Sybil set because they have direct incoming trust from outside the collusion.

  In conclusion, we have delivered on our promise of a Sybil-resilient decentralized financial trust system with invariant
  risk for purchases.

