\section{Εισαγωγή}
  Οι αποκεντρωμένες αγορές μπορούν να κατηγοριοποιηθούν ως κεντρικές και αποκεντρωμένες. Ένα παράδειγμα για κάθε κατηγορία
  είναι το \href{http://www.ebay.com}{\textlatin{ebay}} και το \href{https://openbazaar.org/}{\textlatin{OpenBazaar}}. Ο
  κοινός παρονομαστής των καθιερωμένων διαδικτυακών αγορών είναι το γεγονός ότι η φήμη κάθε πωλητή και πελάτη εκφράζεται κατά
  κανόνα με τη μορφή αστεριών και κριτικών των χρηστών, ορατές σε όλο το δίκτυο.

  Ο στόχος μας είναι να δημιουργήσουμε ένα σύστημα φήμης για αποκεντρωμένες αγορές όπου η εμπιστοσύνη που ο κάθε χρήστης δίνει
  στους υπόλοιπους είναι ποσοτικοποιήσιμη με νομισματικούς όρους. Η κεντρική παραδοχή που χρησιμοποιείται σε όλο το μήκος της
  παρούσας εργασίας είναι ότι η εμπιστοσύνη είναι ισοδύναμε με τον κίνδυνο, ή η θέση ότι η \textit{εμπιστοσύνη} της $Alice$
  προς το χρήστη $Charlie$ ορίζεται ως το \textit{μέγιστο χρηματικό ποσό} που η $Alice$ μπορεί να χάσει όταν ο $Charlie$ είναι
  ελεύθερος να διαλέξει όποια στρατηγική θέλει. Για να υλοποιήσουμε αυτή την ιδέα, θα χρησιμοποιήσουμε τις \textit{πιστωτικές
  γραμμές} όπως προτάθηκαν από τον \textlatin{Washington Sanchez} \cite{loc}. Η $Alice$ συνδέεται στο δίκτυο όταν εμπιστεύεται
  ενεργητικά ένα συγκεκριμένο χρηματικό ποσό σε έναν άλλο χρήστη, για παράδειγμα το φίλο της τον $Bob$. Αν ο $Bob$ έχει ήδη
  εμπιστευθεί ένα χρηματικό ποσό σε έναν τρίτο χρήστη, τον $Charlie$, τότε η $Alice$ εμπιστεύεται έμμεσα τον $Charlie$ αφού αν
  ο τελευταίος ήθελε να παίξει άδικα, θα μπορούσε να έχει κλέψει ήδη τα χρήματα που του εμπιστεύθηκε ο $Bob$. Θα δούμε
  αργότερα ότι η $Alice$ μπορεί τώρα να εμπλακεί σε οικονομική δραστηριότητα με τον $Charlie$.

  Για να υλοποιήσουμε τις πιστωτικές γραμμές, θα χρησιμοποιήσουμε το \textlatin{Bitcoin} \cite{bitcoin}, ένα αποκεντρωμένο
  κρυπτονόμισμα που διαφέρει από τα συμβατικά νομίσματα γιατί δεν βασίζεται σε αξιόπιστους τρίτους. Όλες οι συναλλαγές
  δημοσιεύονται σε ένα αποκεντρωμένο <<λογιστικό βιβλίο>>, το \textlatin{block\-chain}. Κάθε συναλλαγή παίρνει κάποια
  νομίσματα ως είσοδο και παράγει ορισμένα νομίσματα ως έξοδο. Αν η έξοδος μιας συναλλαγής δεν συνδέεται στην είσοδο μιας
  άλλης, τότε η έξοδος αυτή ανήκει στο \textlatin{UTXO}, το σύνολο των αξόδευτων εξόδων συναλλαγών. Διαισθητικά, το
  \textlatin{UTXO} περιέχει όλα τα αξόδευτα νομίσματα.
  \medskip \ \\
  \subimport{common/figures/}{simpleexample.tikz} \smallskip \ \\
  Προτείνουμε ένα νέο είδος πορτοφολιού όπου τα νομίσματα δεν έχουν αποκλειστικό ιδιοκτήτη, αλλά τοποθετούνται σε
  μοιραζόμενους λογαρια\-σμούς που υλοποιούνται μέσω των 1-από-2 \textlatin{multisig}, μια κατασκευή του \textlatin{bit\-coin}
  που επιτρέπει έναν από δύο προκαθορισμένους χρήστες να ξοδέψουν τα νομίσματα που περιέχονται σε έναν κοινό λογαριασμό
  \cite{masteringbitcoin}. Θα χρησιμοποιήσουμε το συμβολισμό 1/$\{Alice, Bob\}$ για να αναπαραστήσουμε ένα 1-από-2
  \textlatin{multisig} που μπορεί να ξοδευτεί είτε από την $Alice$, είτε από τον $Bob$. Με αυτό το συμβολισμό, η σειρά των
  ονομάτων δεν έχει σημασία, εφ' όσον οποιοσδήποτε από τους δύο χρήστες μπορεί να ξοδέψει τα νομίσματα. Ωστόσο, έχει σημασία
  ποιος χρήστης καταθέτει τα χρήματα αρχικά στον κοινό λογαριασμό -- αυτός ο χρήστης διακινδυνεύει τα νομίσματά του.

  Η προσέγγισή μας αλλάζει την εμπειρία του χρήστη κατά έναν διακριτικό αλλά και δραστικό τρόπο. Ο χρήστης δεν πρέπει να
  βασίζει πια την εμπιστοσύνη του προς ένα κατάστημα σε αστέρια ή κριτικές που δεν εκφράζονται με οικονομικές μονάδες. Μπορεί
  απλά να συμβουλευθεί το πορτοφόλι της για να αποφασίσει αν το κατάστημα είναι αξιόπιστο και, αν ναι, μέχρι ποια αξία,
  μετρημένη σε \textlatin{bitcoin}. Το σύστημα αυτό λειτουργεί ως εξής: Αρχικά η $Alice$ μεταφέρει τα χρήματά της από το
  ιδιωτικό της \textlatin{bitcoin} πορτοφόλι σε 0-από-2 διευθύνσεις \textlatin{multisig} μοιραζόμενες με φίλους που
  εμπιστεύεται άνετα. Αυτό καλείται άμεση εμπιστοσύνη. Το σύστημά μας δεν ενδιαφέρεται για τον τρόπο με τον οποίο οι παίκτες
  καθορίζουν ποιος είναι αξιόπιστος γι' αυτές τις απ' ευθείας 1-από-2 καταθέσεις. Αυτό το αμφιλεγόμενο είδος εμπιστοσύνης
  περιορίζεται στην άμεση γειτονιά κάθε παίκτη. Η έμμεση εμπιστοσύνη προς άγνωστους χρήστες υπολογίζεται από έναν
  ντετερμινιστικό αλγόριθμο. Συγκριτικά, συστήματα με αντικειμενικές αξιολογήσεις δε διαχωρίζουν τους γείτονες από τους
  υπόλοιπους χρήστες, προσφέροντας έτσι αμφιλεγόμενες ενδείξεις εμπιστοσύνης για όλους.

  Ας υποθέσουμε ότι η $Alice$ βλέπει τα προϊόντα του πωλητή $Charlie$. Αντί για τα αστέρια του $Charlie$, η $Alice$ θα δει ένα
  θετικό αριθμό που υπολογίζεται από το πορτοφόλι της και αναπαριστά τη μέγιστη χρηματική αξία που η $Alice$ μπορεί να
  πληρώσει με ασφάλεια για να ολοκληρώσει μια αγορά από τον $Charlie$. Αυτή η αξία, γνωστή ως έμμεση εμπιστοσύνη, υπολογίζεται
  με το θεώρημα Εμπιστοσύνης -- Ροής (\ref{trustflow}). Σημειώστε ότι η έμμεση εμπιστοσύνη προς κάποιο χρήστη δεν είναι ενιαία
  αλλά υποκειμενική. Κάθε χρήστης βλέπει μια ιδιαίτερη έμμεση εμπιστοσύνη που εξαρτάται από την τοπολογία του δικτύου. Η
  έμμεση εμπιστοσύνη που εμφανίζεται από το σύστημά μας διαθέτει την ακόλουθη επιθυμητή ιδιότητα ασφαλείας: Αν η $Alice$
  πραγματοποιήσει μια αγορά από τον $Charlie$, τότε εκτίθεται το πολύ στον ίδιο κίνδυνο στον οποίον εκτιθόταν πριν την αγορά.
  Ο υπαρκτός εθελούσιος κίνδυνος είναι ακριβώς εκείνος που η $Alice$ έπαιρνε μοιραζόμενη τα νομίσματά της με τους αξιόπιστους
  φίλους της. Αποδεικνύουμε το αποτέλεσμα αυτό στο θεώρημα Αμετάβλητου Κινδύνου (\ref{riskinv}). Προφανώς δε θα είναι ασφαλές
  για την $Alice$ να αγοράσει οτιδήποτε από τον $Charlie$ ή από οποιονδήποτε άλλο πωλητή αν δεν έχει ήδη εμπιστευθεί καθόλου
  χρήματα σε κανέναν άλλο χρήστη.

  Βλέπουμε ότι στο \textlatin{Trust Is Risk} τα χρήματα δεν επενδύονται τη στιγμή της αγοράς και κατ' ευθείαν στον πωλητή,
  αλλά σε μια προγενέστερη χρονική στιγμή και μόνο προς άτομα που είναι αξιόπιστα για λόγους εκτός παιχνιδιού. Το γεγονός ότι
  το σύστημα αυτό μπορεί να λειτουργήσει με έναν εξ ολοκλήρου αποκεντρωμένο τρόπο θα γίνει σαφές στις επόμενες ενότητες. Θα
  αποδείξουμε το αποτέλεσμα αυτό στο θεώρημα \textlatin{Sybil} Αντίσταστης (\ref{sybil}).

  Κάνουμε τη σχεδιαστική επιλογή ότι κανείς μπορεί να εκφράζει την εμπιστοσύνη του μεγιστικά με όρους του διαθέσιμού του
  κεφαλαίου. Έτσι, ένας φτωχός παίκτης δεν μπορεί να διαθέσει πολλή άμεση εμπιστοσύνη στους φίλους τους ανεξαρτήτως του πόσο
  αξιόπιστοι είναι. Από την άλλη, ένας πλούσιος παίκτης μπορεί να εμπιστευθεί ένα μικρό μέρος των χρημάτων της σε κάποιον
  παίκτη που δεν εμπιστεύεται εκτενώς και παρ' όλα αυτά να εμφανίζει περισσότερη άμεση εμπιστοσύνη από τον φτωχό παίκτη του
  προηγούμενου παραδείγματος. Δεν υπάρχει άνω όριο στην εμπιστοσύνη. Κάθε παίκτης περιορίζεται μόνο από τα χρήματά του. Έτσι
  εκμεταλλευόμαστε την παρακάτω αξιοσημείωτη ιδιότητα του χρήματος: Το ότι κανονικοποιεί τις υποκειμενικές ανθρώπινες
  επιθυμίες σε αντικειμενική αξία.

  Υπάρχουν διάφορα κίνητρα για να συνδεθεί ένας χρήστης στο δίκτυο αυτό. Πρώτον, έχει πρόσβαση σε καταστήματα που αλλιώς θα
  ήταν απρόσιτα. Επίσης, δύο φίλοι μπορούν να επισημοποιήσουν την αλληλοεμπιστοσύνη τους εμπιστεύοντας το ίδιο ποσό ο ένας
  στον άλλο. Μια μεγάλη εταιρεία που πραγματοποιεί συχνά συμβάσεις υπεργολαβίας με άλλες εταιρείες μπορεί να εκφράσει την
  εμπιστοσύνη της προς αυτές. Μια κυβέρνηση μπορεί να εμπιστευθεί άμεσα τους πολίτης της με χρήματα και να τους αντιμετωπίσει
  με ένα ανάλογο νομικό οπλοστάσιο αν αυτοί κάνουν ανεύθυνη χρήση της εμπιστοσύνης αυτής. Μια τράπεζα μπορεί να προσφέρει
  δάνεια ως εξερχόμενες και να χειρίζεται τις καταθέσεις ως εισερχόμενες άμεσες εμπιστοσύνες. Τέλος, το δίκτυο μπορεί να
  ειδωθεί ως ένα πεδίο επένδυσης και κερδοσκοπίας αφού αποτελεί ένα εντελώς νέο πεδίο οικονομικής δραστηριότητας.

  Είναι αξισημείωτο το ότι το ίδιο φυσικό πρόσωπο μπορεί να διατη\-ρεί πολλαπλές ψευδώνυμες ταυτότητες στο ίδιο δίκτυο
  εμπιστοσύνης και ότι πολλά ανεξάρτητα δίκτυα εμπιστοσύνης διαφορετικών σκοπών μπορούν να συνυπάρχουν. Από την άλλη, η ίδια
  ψευδώνυμη ταυτότητα μπορεί να χρησιμοποιηθεί για να αναπτύξει σχέσεις εμπιστοσύνης σε διαφορετικά περιβάλλοντα.
