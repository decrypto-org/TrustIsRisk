\phantomsection
\addcontentsline{toc}{subsection}{Ορισμός Συνεργασίας}
\begin{definitiongr}{Συνεργασία}
  Έστω $\mathcal{G}$ γράφος παιχνιδιού. Έστω $\mathcal{B} \subset \mathcal{V}$ ένα διεφθαρμένο σύνολο και $\mathcal{C} \subset
  \mathcal{V}$ ένα \textlatin{Sybil} σύνολο, ελεγχόμενα και τα δύο από την \textlatin{Eve}. Το διατεταγμένο ζεύγος
  $\left(\mathcal{B}, \mathcal{C}\right)$ αποκαλείται συνεργασία και ελέγχεται εξ ολοκλήρου από μία μοναδική οντότητα στο
  φυσικό κόσμο. Από παιγνιοθεωρητική οπτική, οι παίκτες $\mathcal{V} \setminus (\mathcal{B} \cup \mathcal{C})$ εκλαμβάνουν τη
  συνεργασία ως ανεξάρτητες παίκτες με διαφορετική στρατηγική η καθεμία, ενώ στην πραγματικότητα υπάγονται όλες σε μία
  μοναδική στρατηγική που ορίζεται από την οντότητα που ελέγχει, την \textlatin{Eve}.
\end{definitiongr}
