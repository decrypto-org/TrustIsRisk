\begin{sepproof}{Απόδειξη του Θεωρήματος~\ref{convergence}: Σύγκλιση Εμπιστοσύνης} \ \\
\label{convergenceproof}
  Πρώτα απ' όλα, μετά το γύρο $j_0$ η κακιά παίκτης $E$ θα παίζει πάντα <<πάσο>> γιατί έχει ήδη μηδενίσει την εισερχόμενη και
  εξερχόμενη εμπιστοσύνη στο γύρο $j_0$, η κακιά στρατηγική δεν περιέχει καμία περίπτωση κατά την οποία η άμεση εμπιστοσύνη να
  αυξάνεται ή κατά την οποία η κακιά παίκτης ξεκινά να εμπιστεύεται άμεσα κάποια άλλη παίκτη και οι άλλες παίκτες δεν
  ακολουθούν κάποια στρατηγική στην οποία να μπορούν να επιλέξουν την πράξη $Add\left(y, E\right)$. Το ίδιο ισχύει για την
  παίκτη $A$ γιατί ακολουθεί την αδρανή στρατηγική. Όσον αφορά τους υπόλοιπους παίκτες, θεωρήστε το Μεταβατικό Παιχνίδι. Όπως
  βλέπουμε στις γραμμές~\ref{trsteallossinit} και~\ref{trsteallossincrease} -~\ref{trsteallossdecrease}, είναι
  \begin{equation*}
    \forall j, \sum\limits_{v \in \mathcal{V}_j}Loss_v = in_{E, j_0-1} \enspace.
  \end{equation*}
  Με άλλα λόγια, η συνολική ζημία είναι σταθερή και ίση με τη συνολική αξία που έκλεψε η $E$. Επίσης, όπως μπορούμε να δούμε
  στις γραμμές~\ref{trstealsadinit} και~\ref{trstealtrueentersad}, οι οποίες είναι οι μόνες γραμμές όπου το σύνολο των
  Λυπημένων μεταβάλλεται, μόλις μία παίκτης μπει στο Λυπημένο σύνολο, είναι αδύνατο να βγει από αυτό. Επίσης βλέπουμε ότι οι
  παίκτες στα Λυπημένο και Χαρούμενο σύνολα πάντα παίζουν <<πάσο>>. Θα δείξουμε τώρα ότι τελικά το σύνολο Θυμωμένων θα
  αδειάσει, ή ισοδύναμα ότι τελικά κάθε παίκτης θα παίζει <<πάσο>>. Ας υποθέσουμε ότι είναι δυνατό να έχουμε άπειρους γύρους
  κατά τους οποίους οι παίκτες δεν επιλέγουν να παίξουν <<πάσο>>. Γνωρίζουμε ότι ο αριθμός των κόμβων είναι πεπερασμένος,
  συνεπώς αυτό είναι δυνατό μόνο αν
  \begin{equation*}
    \exists j': \forall j \geq j', |Angry_j \cup Happy_j| = c > 0 \wedge Angry_j \neq \emptyset \enspace.
  \end{equation*}
  Ο ισχυρισμός αυτός είναι έγκυρος γιατί ο συνολικός αριθμός Θυμωμένων και Χαρούμενων παικτών δεν μπορεί να αυξηθεί γιατί
  καμία παίκτης δεν αποχωρεί από το σύνολο Λυπημένων και αν μειωνόταν, θα έφτανε τελικά το 0. Αφού $Angry_j \neq \emptyset$,
  κάποια παίκτης $v$ που δε θα παίξει <<πάσο>> θα επιλεγεί τελικά για να παίξει. Σύμφωνα με το Μεταβατικό Παιχνίδι, η $v$ είτε
  θα μηδενίσει την εισερχόμενη άμεση εμπιστοσύνη της και θα μπει στο σύνολο των Λυπημένων (γραμμή~\ref{trstealtrueentersad}),
  το οποίο αντικρούεται στο $|Angry_j \cup Happy_j| = c$, ή θα κλέψει αρκετή αξία ώστε να μπει στο σύνολο Χαρούμενων, δηλαδή η
  $v$ θα πετύχει $Loss_{v, j} = 0$. Ας υποθέσουμε ότι έχει κλέψει $m$ παίκτες. Εκείνες, στο γύρο τους, θα κλέψουν συνολική
  αξία τουλάχιστον ίση με την αξία που κλάπηκε από την $v$ (αφού δεν μπορούν να γίνουν Λυπημένες, όπως εξηγήθηκε νωρίτερα).
  Ωστόσο, αυτο σημαίνει ότι, αφού η συνολική αξία που κλέβεται δε θα μειωθεέι και οι γύροι που αυτό θα συμβεί είναι άπειροι,
  οι παίκτες πρέπει να κλέψουν ένα άπειρο ποσό αξίας, το οποίο είναι αδύνατο γιατί οι άμεσες εμπιστοσύνες είναι πεπερασμένες
  σε αριθμό και αξία. Πιο συγκεκριμένα, έστω $j_1$ ένας γύρος στον οποίο επιλέγεται ένας συντηρητικός παίκτης και
  \begin{equation*}
    \forall j \in \mathbb{N}, DTr_j = \sum\limits_{w,w' \in \mathcal{V}}DTr_{w \rightarrow w', j} \enspace.
  \end{equation*}
  Επίσης, χωρίς βλάβη της γενικότητας, υποθέτουμε ότι
  \begin{equation*}
    \forall j \geq j_1, out_{A, j} = out_{A, j_1} \enspace.
  \end{equation*}
  Στο γύρο $j_1$, η $v$ κλέβει
  \begin{equation*}
    St = \sum\limits_{i=1}^{m}y_i \enspace.
  \end{equation*}
  Θα δείξουμε με χρήση επαγωγής ότι
  \begin{equation*}
    \forall n \in \mathbb{N}, \exists j_n \in \mathbb{N} : DTr_{j_n} \leq DTr_{j_1-1} - nSt \enspace.
  \end{equation*}

  Επαγωγική βάση: Ισχύει ότι
  \begin{equation*}
    DTr_{j_1} = DTr_{j_1-1} - St \enspace.
  \end{equation*}
  Τελικά υπάρχει γύρος $j_2$ που κάθε παίκτης στο $N^{-}(v)_{j-1}$ θα έχει παίξει. Τότε ισχύει ότι
  \begin{equation*}
    DTr_{j_2} \leq DTr_{j_1} - St = DTr_{j_1-1} - 2St \enspace,
  \end{equation*}
  αφού όλες οι παίκτες στο $N^{-}(v)_{j-1}$ ακολουθούν τη συντηρητική στρατηγική, εκτός της $A$, από την οποία δεν έχει κλαπεί
  τίποτα λόγω της υπόθεσης.

  Επαγωγική υπόθεση: Υποθέτουμε ότι
  \begin{equation*}
    \exists k > 1 : j_k > j_{k-1} > j_1 \Rightarrow DTr_{j_k} \leq DTr_{j_{k-1}} - St \enspace.
  \end{equation*}

  Επαγωγικό βήμα: Υπάρχει ένα υποσύνολο των Θυμωμένων παικτών, $S$, από τους οποίους έχει κλαπεί τουλάχιστον $St$ συνολική
  αξία μεταξύ των γύρων $j_{k-1}$ και $j_k$, συνεπώς υπάρχει γύρος $j_{k+1}$ τέτοιος ώστε όλες οι παίκτες στο $S$ να έχουν
  παίξει και συνεπώς
  \begin{equation*}
    DTr_{j_{k+1}} \leq DTr_{j_k} - St \enspace.
  \end{equation*}
  Δείξαμε με επαγωγή ότι
  \begin{equation*}
    \forall n \in \mathbb{N}, \exists j_n \in \mathbb{N} : DTr_{j_n} \leq DTr_{j_1-1} - nSt \enspace.
  \end{equation*}
  Ωστόσο
  \begin{equation*}
    DTr_{j_1-1} \geq 0 \wedge St > 0 \enspace,
  \end{equation*}
  συνεπώς
  \begin{equation*}
    \exists n' \in \mathbb{N} : n'St > DTr_{j_1-1} \Rightarrow DTr_{j_{n'}} < 0 \enspace.
  \end{equation*}
  Έχουμε άτοπο γιατί
  \begin{equation*}
    \forall w,w' \in \mathcal{V}, \forall j \in \mathbb{N}, DTr_{w \rightarrow w', j} \geq 0 \enspace,
  \end{equation*}
  Συνεπώς τελικά $Angry = \emptyset$ και όλοι παίζουν <<πάσο>>.
\end{sepproof}
