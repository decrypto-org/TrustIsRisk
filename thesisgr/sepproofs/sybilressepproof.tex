\begin{sepproof}{Απόδειξη του Θεωρήματος~\ref{sybil}: \textlatin{Sybil} Αντίσταση} \ \\
\label{sybilproof}
  Έστω $\mathcal{G}_1$ γράφος παιχνιδιού που ορίζεται ως εξής:
  \begin{equation*}
    \mathcal{V}_1 = \mathcal{V} \cup \{T_1\} \enspace,
  \end{equation*}
  \begin{equation*}
    \mathcal{E}_1 = \mathcal{E} \cup \{(v, T_1) : v \in \mathcal{B} \cup \mathcal{C}\} \enspace,
  \end{equation*}
  \begin{equation*}
    \forall v,w \in \mathcal{V}_1 \setminus \{T_1\}, DTr^1_{v \rightarrow w} = DTr_{v \rightarrow w} \enspace,
  \end{equation*}
  \begin{equation*}
    \forall v \in \mathcal{B} \cup \mathcal{C}, DTr^1_{v \rightarrow T_1} = \infty \enspace,
  \end{equation*}
  όπου η $DTr_{v \rightarrow w}$ είναι η άμεση εμπιστοσύνη από την $v$ στην $w$ στον $\mathcal{G}$ και η $DTr^1_{v \rightarrow
  w}$ είναι η άμεση εμπιστοσύνη από την $v$ στην $w$ στον $\mathcal{G}_1$. \\
  Έστω επίσης $\mathcal{G}_2$ ο παράγωγος γράφος που προκύπτει από τον $\mathcal{G}_1$ αν αφαιρέσουμε το σύνολο
  \textlatin{Sybil}, $\mathcal{C}$. Μετονομάζουμε τη $T_1$ σε $T_2$ και ορίζουμε $\mathcal{L} = \mathcal{V} \setminus
  \left(\mathcal{B} \cup \mathcal{C}\right)$ ως το σύνολο των τίμιων παικτών για να διευκολύνουμε την κατανόηση.
  \subimport{common/figures/}{sybilres.tikz}
  Σύμφωνα με το Θεώρημα~\ref{trustmany},
  \begin{equation}
  \label{trmaxflow}
    Tr_{A \rightarrow \mathcal{B} \cup \mathcal{C}} = maxFlow_1\left(A, T_1\right) \wedge
    Tr_{A \rightarrow \mathcal{B}} = maxFlow_2\left(A, T_2\right) \enspace.
  \end{equation}
  Θα δείχουμε ότι το $MaxFlow$ του καθενός από τους δύο γράφους μπορεί να χρησιμοποιηθεί για να κατασκευαστεί μία έγκυρη ροή
  ίσης τιμής για τον άλλο γράφο. Η ροή $X_1 = MaxFlow\left(A, T_1\right)$ μπορεί να χρησιμοποιηθεί για να κατασκευαστεί μία
  έγκυρη ροή ίσης τιμής για το δεύτερο γράφο αν θέσουμε
  \begin{align*}
    \forall v \in \mathcal{V}_2 \setminus \mathcal{B}, \forall w \in \mathcal{V}_2&, x_{vw,2} = x_{vw,1} \enspace, \\
    \forall v \in \mathcal{B}&, x_{vT_2,2} = \sum\limits_{w \in N^{+}_1\left(v\right)}x_{vw,1} \enspace, \\
    \forall v,w \in \mathcal{B}&, x_{vw,2} = 0 \enspace.
  \end{align*}
  Έτσι
  \begin{equation*}
    maxFlow_1\left(A, T_1\right) \leq maxFlow_2\left(A, T_2\right)
  \end{equation*}
  Ομοίως, η ροή $X_2 = MaxFlow(A, T_2)$ είναι μία έγκυρη ροή για τον $\mathcal{G}_1$ γιατί ο $\mathcal{G}_2$ είναι ένας
  παράγωγος υπογράφος του $\mathcal{G}_1$. Συνεπώς
  \begin{equation*}
    maxFlow_1\left(A, T_1\right) \geq maxFlow_2\left(A, T_2\right)
  \end{equation*}
  Συμπεραίνουμε ότι
  \begin{equation}
  \label{eqmaxflows}
    maxFlow\left(A, T_1\right) = maxFlow\left(A, T_2\right) \enspace,
  \end{equation}
  άρα από το~(\ref{trmaxflow}) και το~(\ref{eqmaxflows}) το θεώρημα ισχύει.
\end{sepproof}
