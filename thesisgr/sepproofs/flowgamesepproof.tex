\begin{sepproof}{Απόδειξη του Λήμματος~\ref{maxflowgame}: Οι μέγιστες ροές είναι Μεταβατικά Παιχνίδια} \ \\
\label{maxflowgameproof}
  Υποθέτουμε ότι ο γύρος του $\mathcal{G}$ είναι ο 0. Με άλλα λόγια, $\mathcal{G} = \mathcal{G}_0$. Έστω $X =
  \{x_{vw}\}_{\mathcal{V} \times \mathcal{V}}$ οι ροές που επιστρέφονται από τον $MaxFlow\left(A, B\right)$. Για κάθε γράφο
  $G$ υπάρχει απόδοση $MaxFlow$ που να είναι κατευθυνόμενος ακυκλικός γράφος (ΚΑΓ). Αυτό μπορεί να αποδειχθεί εύκολα με χρήση
  του θεωρήματος Αποσύνθεσης Ροής \cite{amo}, το οποίο δηλώνει ότι κάθε ροή μπορεί να ειδωθεί ως ένα πεπερασμένο σύνολο
  μονοπατιών από τον $A$ στον $B$ και κύκλων, ο καθένας εκ των οποίων εχει μία συγκεκριμένη ροή. Εκτελούμε το $MaxFlow\left(A,
  B\right)$ και εφαρμόζουμε το προαναφερθέν θεώρημα. Οι κύκλοι δεν επηρεάζουν το $MaxFlow\left(A, B\right)$, συνεπώς μπορούμε
  να αφαιρέσουμε τις ροές αυτές. Η προκύπτουσα ροή είναι ένα $MaxFlow\left(A, B\right)$ χωρίς κύκλους, συνεπώς είναι ένας ΚΑΓ.
  Εκτελώντας τοπολογική ταξινόμηση σε αυτόν τον ΚΑΓ, παίρνουμε μία ολική διάταξη των κόμβων του έτσι που $\forall v, w \in
  \mathcal{V} : v < w \Rightarrow x_{wv} = 0$ \cite{clrs}. Θέτοντάς το διαφορετικά, δεν υπάρχει ροή από μεγαλύτερους προς
  μικρότερους κόμβους. Ο $B$ είναι μέγιστος αφού είναι η καταβόθρα και συνεπώς δεν έχει εξερχόμενη ροή προς άλλους κόμβους και
  ο $A$ είναι ελάχιστος γιατί είναι η πηγή και συνεπώς δεν έχει καθόλου εισερχόμενη ροή από άλλους κόμβους. Η επιθυμητή
  εκτέλεση του Μεταβατικού Παιχνιδιού θα διαλέξει παίκτες ακολουθώντας την ολική διάταξη αντίστροφα, ξεκινώντας από την παίκτη
  $B$. Παρατηρού\-με ότι $\forall v \in \mathcal{V} \setminus \{A, B\}, \sum\limits_{w \in \mathcal{V}}x_{wv} = \sum\limits_{w
  \in \mathcal{V}}x_{vw} \leq maxFlow\left(A, B\right) \leq in_{B, 0}$. Η παίκτης $B$ θα ακολουθήσει μία τροποποιημένη κακιά
  στρατηγική στην οποία κλέβει αξία ίση με τη συνολική εισερχόμενη ροή, όχι τη συνολική εισερχόμενη εμπιστοσύνη. Έστω $j_2$ ο
  πρώτος γύρος στον οποίο επιλεγεται η $A$. Θα δείξουμε χρησιμοποιώντας ισχυρή επαγωγή ότι υπάρχει σύνολο έγκυρων πράξεων για
  κάθε παίκτη ανάλογα με τη στρατηγική της τέτοιο ώστε στο τέλος του γύρου $j$ η αντίστοιχη παίκτης $v = Player\left(j\right)$
  θα έχει κλέψει αξία $x_{wv}$ από κάθε μέσα γείτονα $w$.

  Επαγωγική βάση: Στο γύρο 1, η $B$ κλέβει αξία ίση με $\sum\limits_{w \in \mathcal{V}}x_{wB}$, ακολουθώντας την τροποποιημένη
  κακιά στρατηγική.
  \begin{equation*}
    Turn_1 = \bigcup\limits_{v \in N^{-}\left(B\right)_0}\{Steal\left(x_{vB}, v\right)\}
  \end{equation*}

  Επαγωγική υπόθεση: Έστω $k \in [j_2 - 2]$. Υποθέτουμε ότι $\forall i \in [k]$, υπάρχει ένα έγκυρο σύνολο πράξεων, $Turn_i$,
  εκτελεσμένων από την $v = Player\left(i\right)$ τέτοιο ώστε η $v$ να κλέψει από κάθε παίκτη $w$ αξία ίση με $x_{wv}$.
  \begin{equation*}
    \forall i \in [k], Turn_i = \bigcup\limits_{w \in N^{-}\left(v\right)_{i-1}}\{Steal\left(x_{wv}, w\right)\}
  \end{equation*}

  Επαγωγικό βήμα: Έστω $j = k + 1, v = Player\left(j\right)$. Αφού όλες οι παίκτες που είναι μεγαλύτερες από την $v$ στην
  ολική διάταξη έχουν ήδη παίξει και όλες τους έχουν κλέψει αξία ίση με την εισερχόμενη ροή τους, συμπεραίνουμε ότι από τη $v$
  έχει κλαπεί αξία ίση με $\sum\limits_{w \in N^{+}\left(v\right)_{j-1}}x_{vw}$. Αφού αυτή είναι η πρώτη φορά που η $v$
  παίζει, $\forall w \in N^{-}\left(v\right)_{j-1}, DTr_{w \rightarrow v, j-1} = DTr_{w \rightarrow v, 0} \geq x_{wv}$,
  συνεπώς η $v$ μπορεί να διαλέξει τον εξής γύρο:
  \begin{equation*}
    Turn_j = \bigcup\limits_{w \in N^{-}\left(v\right)_{j-1}}\{Steal\left(x_{wv}, w\right)\}
  \end{equation*}
  Επιπλέον, αυτός ο γύρος ικανοποιεί τη συντηρητική στρατηγική αφού
  \begin{equation*}
    \sum\limits_{w \in N^{-}\left(v\right)_{j-1}}x_{wv} = \sum\limits_{w \in N^{+}\left(v\right)_{j-1}}x_{vw} \enspace.
  \end{equation*}
  Άρα ο $Turn_j$ είναι ένας έγκυρος γύρος για τη συντηρητική παίκτη $v$.

  Δείξαμε ότι στο τέλος του γύρου $j_2 - 1$, η παίκτης $B$ και όλες οι συντηρητικές παίκτες θα έχουν κλέψει αξία ακριβώς ίση
  με την εισερχόμενη ροή, συνεπώς από την $A$ θα έχει κλαπεί αξία ίση με την εξερχόμενη ροή της, η οποία είναι
  $maxFlow\left(A, B\right)$. Αφού δε μένει άλλη Θυμωμένη παίκτης, ο $j_2$ είναι γύρος σύγκλισης, συνεπώς $Loss_{A, j_2} =
  Loss_A$. Μπορούμε επίσης να δούμε ότι αν η $B$ είχε διαλέξει την κανονική κακιά στρατηγική, οι πράξεις που περιγράφηκαν θα
  εξακολουθούσαν να είναι έγκυρες με απλή προσθήκη επιπλέον πράξεων $Steal\left(\right)$, συνεπώς η $Loss_A$ θα αυξανόταν
  περαιτέρω. Αυτό αποδεικνύει το λήμμα.
\end{sepproof}
